\documentclass[12pt,a4paper]{article}
\usepackage[utf8]{inputenc}
\usepackage{paralist}

%Tittel på dokumentet skal være "ukesreferat-uke[nåværende uke].tex"

\begin{document}
\title{
Ukereferat\\
Gruppe 15\\
Uke 9
}
\author{Kristoffer Jensen}
\date{04.03.14}
\maketitle

\section*{Arbeidsplan}

%Mål vi bestemte i forrige uke for nåværende uke
\subsection*{Mål fra forrige uke:}
\begin{compactitem}
	\item Jobbe videre med å få serveren til å oppdatere databasen og ta i mot informasjon fra ethernutkortet. (Mikael)
	\item Jobbe videre med programmering av ethernutkortet. (Kristoffer)
	\item Anskaffe komponenter til sikkerhetskresten til temperatur- og luftrykksensor. Lage et nytt utkast av hele kretsen til de to sensorene som inkluderer sikkerhetskretsen.
	\item De som får tid skriver videre på hovedrapporten. (Alle)
\end{compactitem}

%Punktliste med det vi faktisk fikk gjort denne uken og litt om de forskjellige oppgavene. Ta med eventuelle mål vi ikke fikk gjort, og hvis vi gjorde mer enn det vi planla osv.
\subsection*{Arbeidsoppgaver for denne uken:}
\begin{compactitem}
	\item Jobbe videre med å få serveren til å oppdatere databasen og ta i mot informasjon fra ethernutkortet. (Mikael)
	Programmet for serveren er nesten ferdig. Trenger eventuelt mer jobb senere.
	\item Jobbe videre med programmering av ethernutkortet. (Kristoffer)
	Fått ethernut til å kommunisere med server.
	\item Anskaffe komponenter til sikkerhetskresten til temperatur- og luftrykksensor. Lage et nytt utkast av hele kretsen til de to sensorene som inkluderer sikkerhetskretsen.
	Har fått tak i komponenter fra Geir.
	\item De som får tid skriver videre på hovedrapporten. (Alle)
\end{compactitem}

%Mål vi har satt oss for neste uke, og evt. litt om de.
\subsection*{Arbeidsoppgaver for kommende uke:}
\begin{compactitem}
	\item Jobbe videre med programmering av ethernutkortet.
	\item Jobbe videre med kretsene mellom ethernut og sensorer.
	\item Arbeide med første versjon av hoveddokument som skal leveres 14.03
\end{compactitem}
\end{document}
