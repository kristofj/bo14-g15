\documentclass[12pt,a4paper]{article}
\usepackage[utf8]{inputenc}
\usepackage{paralist}

%Tittel på dokumentet skal være "ukesreferat-uke[nåværende uke].tex"

\begin{document}
\title{
Ukereferat\\
Gruppe 15\\
Uke 5
}
\author{Kristoffer Jensen}
\date{03.02.14}
\maketitle

\section*{Arbeidsplan}

%Mål vi bestemte i forrige uke for nåværende uke
\subsection*{Mål fra forrige uke:}
\begin{compactitem}
        \item Få ferdig sensorliste og levere til arbeidsgiver slik at han kan få bestillt.
        \item Planlegge kretskjema for hardware.
        \item Planlegge hvilken teknologi vi skal bruke for å presentere værdata på hessdalen.org.
\end{compactitem}

%Punktliste med det vi faktisk fikk gjort denne uken og litt om de forskjellige oppgavene. Ta med eventuelle mål vi ikke fikk gjort, og hvis vi gjorde mer enn det vi planla osv.
\subsection*{Arbeidsoppgaver for denne uken:}
\begin{compactitem}
        \item Få ferdig sensorliste og levere til arbeidsgiver slik at han kan få bestillt:
      	Liste over sensorer ble levert til oppdragsgiver.
        \item Planlegge kretskjema for hardware:
        Dette blir dyttet videre til neste uke, da vi vil være sikre på hvilke sensorer vi skal benytte før vi planlegger videre.
        \item Planlegge hvilken teknologi vi skal bruke for å presentere værdata på hessdalen.org:
        
I vårt valg av hvilke teknologier vi skal bruke for å presentere værdata på hessdalen.org, har vi tatt utangspunkt i de vi hadde grei kunskap om fra før. Det vil si JavaScript, PHP, og CGI-basert Python.\\
For å best kunne vurdere hva vi skulle velge, satt vi opp en liste med fordeler og ulemper ved de forskjellige metodene:\\

\subsubsection*{CGI-basert Python}
Fordeler: Den fulle kraften av Python implementert i nettsiden. Mange muligheter, og en stort menge biblioteker tilgjengelig.

Ulemper: Hvis man ikke bruker alternative måter å kjøre CGI på, kan det fort bli treg utførelsetid på serveren. All kode kjøres også på serveren, noe som fører til større belastning.

\subsubsection*{JavaScript}
Fordeler: Laget for webprogrammering! Stor mengde relevante bilioteker for presentering av data (Grafer, diagrammer, osv.) God samhandling med AJAX-teknologi for sømløs lasting av innhold. Veldig utbredt språk, så det er mye dokumentasjon. Kode kjøres av nettleser, noe som fører til lavere belastning på server.

Ulemper: Krever litt mer å debugge, men Firebug (utvidelse til Firefox og Chrome) ordner dette relativt greit.

\subsubsection*{PHP}
Fordeler: Veldig lett syntaks og debugging. Tilgivende språk (det meste fortsetter å kjøre selv om man har kodefeil).

Ulemper: Hatt mange problemer med sikkerhet i sin levetid. All kode kjøres også på serveren, noe som fører til større belastning.\\

På bakgrunn av dette har vi bestemt oss for å presentere data ved å hovedsaklig bruke JavaScript med AJAX-teknologi og biblioteker for grafer, osv. så langt det lar seg gjøre. Kjører vi oss fast, vil det være mulig å lage enkelt moduler i PHP, men vi vil styre helt unna CGI-basert programmering.

\end{compactitem}

%Mål vi har satt oss for neste uke, og evt. litt om de.
\subsection*{Arbeidsoppgaver for kommende uke:}
\begin{compactitem}
	\item Planlegge kretsskjema for hardware.
	\item Planlegge hvordan vi skal beskytte elektronikk mot vær.
\end{compactitem}
\end{document}
