\documentclass[12pt,a4paper]{article}
\usepackage[utf8]{inputenc}
\usepackage{paralist}

%Tittel p� dokumentet skal v�re "ukesreferat-uke[n�v�rende uke].tex"

\begin{document}
\title{
Ukereferat\\
Gruppe 15\\
Uke 8
}
\author{Kristian N. Karlsen}
\date{23.02.14}
\maketitle

\section*{Arbeidsplan}

%M�l vi bestemte i forrige uke for n�v�rende uke
\subsection*{M�l fra forrige uke:}
\begin{compactitem}
	\item Fortsette med programmering av Ethernut (Mikael og Kristoffer).
	\item Sette opp kretsskjema for systemet (Morten og Kristian)
	\item Fortsette med hovedrapporten (Morten og Kristian)	
\end{compactitem}

%Punktliste med det vi faktisk fikk gjort denne uken og litt om de forskjellige oppgavene. Ta med eventuelle m�l vi ikke fikk gjort, og hvis vi gjorde mer enn det vi planla osv.
\subsection*{Arbeidsoppgaver for denne uken:}
\begin{compactitem}
	\item Har f�tt til kommunikasjon mellom ethernutkortet og server. Oppdatert databasen til � innfrie nye krav. (Kristoffer og Mikael)
	\item Har f�tt satt opp kretsskjema for temperatur- og luftfuktighetsensoren, og satt opp til trykksensoren. Mangler � skaffe zenerdiode og motstander til sikkerhetskretsen. (Morten og Kristian)
	\item Har f�tt laget skallet til hovedrapporten (Kristian)
\end{compactitem}

%M�l vi har satt oss for neste uke, og evt. litt om de.
\subsection*{Arbeidsoppgaver for kommende uke:}
\begin{compactitem}
	\item Jobbe videre med � f� serveren til � oppdatere databasen og ta i mot informasjon fra ethernutkortet. (Mikael)
	\item Jobbe videre med programmering av ethernutkortet. (Kristoffer)
	\item Anskaffe komponenter til sikkerhetskresten til temperatur- og luftrykksensor. Lage et nytt utkast av hele kretsen til de to sensorene som inkluderer sikkerhetskretsen.
	\item De som f�r tid skriver videre p� hovedrapporten. (Alle)
\end{compactitem}
\end{document}