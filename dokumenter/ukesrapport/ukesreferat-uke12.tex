\documentclass[12pt,a4paper]{article}
\usepackage[utf8]{inputenc}
\usepackage{paralist}

%Tittel på dokumentet skal være "ukesreferat-uke[nåværende uke].tex"

\begin{document}
\title{
Ukereferat\\
Gruppe 15\\
Uke 12
}
\author{Kristian Norum Karlsen}
\date{24.03.2014}
\maketitle

\section*{Arbeidsplan}

%Mål vi bestemte i forrige uke for nåværende uke
\subsection*{Mål fra forrige uke:}
\begin{compactitem}
	\item Starte arbeid med å programmere fremvisning av data på nettsiden.
	\item Videre programmering av Ethernut mot sensorer
	\item Planlegge hvordan kortet og sensorene skal bygges inn.		
\end{compactitem}

%Punktliste med det vi faktisk fikk gjort denne uken og litt om de forskjellige oppgavene. Ta med eventuelle mål vi ikke fikk gjort, og hvis vi gjorde mer enn det vi planla osv.
\subsection*{Arbeidsoppgaver for denne uken:}
\begin{compactitem}
	\item Bestillt og mottatt strømforsyinger til vindsensorene.
	\item Kommunikasjon mellom BMP180 og ethernut er fungerende og vi får lest av korrekte målinger.
	\item Visualisering av målingene er påbegynt. Vi skal bruke Google charts som ligger under Google visualization API.
\end{compactitem}

%Mål vi har satt oss for neste uke, og evt. litt om de.
\subsection*{Arbeidsoppgaver for kommende uke:}
\begin{compactitem}
	\item Finpusse på koden til ethenruten.
	\item Jobbe videre med Google Charts.
	\item Skrive på hovedrapporten.
	\item Lage overgang mellom strømforsyner og vindsensorene.
	\item Lage kabinett til ethernuten.
\end{compactitem}
\end{document}