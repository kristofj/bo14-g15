\documentclass[11pt,a4paper]{report} 

\usepackage[utf8x]{inputenc} 
\usepackage[norsk]{babel} 
\usepackage{lipsum,paralist}
\usepackage[none]{hyphenat}
\usepackage[colorinlistoftodos]{todonotes}


\begin{document}
\title{
Forprosjektrapport \\
\vspace{2cm}
Prosjektets tittel\\
Gruppe 15
}
\author{
\LARGE 
Mikael Johansen Grimstad, Kristian Norum Karlsen, \\* \LARGE Kristoffer Jensen og Morten Lindstad}
\maketitle

\section*{Prosjektgruppen}

Morten Lindstad(født 10.05.1992) gikk tidligere på elektrofag med studiekompetanse(2008-2011) før han startet på dataningeniørlinja på Høgskolen i Østfold. Dette viste seg å ha flere fordeler med tanke på at flere av faga går igjen med hva han hadde på videregående. Han har alltid hatt en stor interesse for elektronikk og data, alt fra elektriske gitarer til dataspill. Dette er også en fordel fordi han og vil mest sannsynlig ha en enklere forståelse for Bacheloroppgaven.  \\

Mikael Johansen Grimstad(født 23.08.1991) gikk tidligere studiespesialisering ved Frederik II VGS i Fredrikstad(2007-2010). I året som fulgte utførte han verneplikt som Gardist ved Hans Majestet Kongens Garde. Videre valgte å studere Informatikk - Design og utvikling av IT-systemer ved Høgskolen i Østfold fordi han alltid har hatt en stor interesse for elektronikk, data og programmering. \\

Kristoffer Jensen(født 14.11.1991) gikk tidligere studiespesialiserende elektrofag ved Glemmen VGS(2007-2010). I året som fulgte utførte han verneplikt som Luftvernartillerist ved Ørland hovedflystasjon. Videre valgte han bachelorstudiet dataingeniør ved Høgskolen i Østfold da dette virket som en naturlig valg med tanke på hans interesser.\\

Kort beskrivelse av dere selv: Faglig bakgrunn, interesser etc.


\section*{Oppdragsgiver}


Oppdragsgivern for våres bachelorprosjekt er Erling P. Strand. Han er foreleser ved Høgskolen i Østfold. Han underviser i fag som tressfysikk, datakommunikasjon, fysikk/kjemi og datateknikk. Han har tidligere bakgrunn som sivilingeniør og utdannet seg til dette i Trondheim.\\

Han har i tillegg et prosjekt ved siden av han kaller Prosject Hessdalen. Etter det ble oppdaget ukjente lysfenomener i Hessdalen i slutten av 1981 ble Prosject Hessdalen grunnlagt. Det har i senere tid blitt mye mediaoppmerksomhet med "Hessdalsfenomenet". Strand er i dag leder av dette prosjektet. \\

Erling Strand vil hente mest mulig informasjon om disse lysfenomenene ved hjelp av kameraer og sensores og registerer bilder og bevegelse. Han ville ha en studentgruppe fra HiØ som skal sette opp en komplett værstasjon bestående av diverse sensorer i Hessdalen.



\section*{Oppdraget}

Gruppen skal lage og montere to værstasjoner i Hessdalen. Disse værstasjonene skal bestå av en rekke sensorer som er koblet til mikroprosessoren Ethernut 2.1. Mikroprosessorene skal programmeres slik at de lagrer data i en database på HiØ.\\

Disse dataene skal presenteres på nettsiden til prosjekt Hessdalen. Ved hjelp av disse dataene kan oppdragsgiver undersøke om værforhold er en innvirkning på lysfenomenene i Hessdalen. Vi skal måle temperatur, lufttrykk, vindhastighet, vindretning og luftfuktighet.

I denne delen skal dere også ta for dere  tre sentrale aspekter: Formål, leveranser og metode. {\em Formålet} (ofte bare kalt {\em målet}, skal beskrive virkningen av prosjektet på et overordnet plan (f.eks. øke omsetningen i et firma). 
{\em Leveransene} er konkrete resultater (tangibles) som blir produsert underveis (f.eks. programvare med tilhørende brukerdokumentasjon), mao.\ {\em hva} som skal produseres. 
{\em Metoden} er {\em hvordan} formål og leveranser skal oppnås (f.eks. analysere dagens situasjon og designe og utvikle en ny nettbutikk). 
Jo mer teoretisk og ``akademisk'' prosjektet er, jo større vekt må man legge på metoden. Tradisjonelt er det metodiske aspektet relativt nedtonet i et bachelorprosjekt i forhold til et master- eller PhD-prosjekt.
Erfaringsmessig oppfatter studentene dette som en litt fremmed måte å betrakte et prosjekt på, men den er utbredt i både akademia og næringslivet, og gjær det lettere å holde tunga rett i munnen underveis. 

Formålet uttrykkes gjerne som ett hovedmål, og et par-tre delmål som utdyper hovedmålet. Beskrivelsen av et mål starter nesten alltid med et verb.


\subsection*{Formål}

\begin{compactitem}
\item [{\bf Hovedmål}] Måle værdata i Hessdalen og presentere dette på en nettside.
\begin{compactitem}
\item [{\bf  Delmål 1}] Montere to værstasjoner.
\item [{\bf  Delmål 2}] Programmere mikroprosessorene.
\item [{\bf  Delmål 3}] Lage nettside.
\end{compactitem}
\end{compactitem}

\subsection*{Leveranser}

Gruppen skal levere to værstasjoner som skal installeres i Hessdalen. Disse skal henges opp i trær, og skal kobles til hovedstasjonen til prosjektet.  

\subsection*{Metode}



\section*{Prosjektplan}

Prosjektplanen består av et antall veldefinerte aktiviteter. En aktivitet (task), bør ha disse elementene: 

\begin{compactitem}
\item Tittel/nummer og navn
\item Varighet (startdato, sluttdato)
\item Bemanning 
\item Leveranse(r)
\item Forklarende tekst
\end{compactitem}

Prosjektplanen kan f.eks. presenteres på denne måten\footnote{Det kan jo være greit å lage et Gantt-diagram også.}

\begin{compactdesc}

\item [Aktivitetet 1:] Hjemmeside
	\begin{compactitem}
	\item Start: 1/1
	\item Slutt: 10/1
	\item Bemanning: Alle
	\item Leveranse: Ingen leveranse 
	\item Beskrivelse: Her skal det legges ut en kort beskrivelse av prosjektet, alt av dokumentasjon tilhørende 
    	  prosjektet og kontaktinformasjon til gruppemedlemmene legges ut.
    \item Prioritet:
	\end{compactitem}
    
	\item [Aktivitetet 2:] Forprosjektrapport
	\begin{compactitem}
	\item Start: 10/1
	\item Slutt: 17/1
	\item Bemanning Alle
	\item Leveranse: Einar Von Krogh
	\item Beskrivelse: Beskrivelse av prosjektgruppen, oppdragsgiver, oppdraget, formål, leveranser, metode, prosjektplan og gjennomføring.
    \item Prioritet:
	\end{compactitem}
    
    \item [Aktivitetet 3:] Finne sensorer
	\begin{compactitem}
	\item Start: 6/1
	\item Slutt: 19/1
	\item Bemanning: Alle
	\item Leveranse: Oppdragsgiver
	\item Beskrivelse: Finne de ulike sensorene som passer til Ethernut 2.1
    \item Prioritet:
	\end{compactitem}
    
    \item [Aktivitetet 4:] Opprette databasen
	\begin{compactitem}
	\item Start: 20/1
	\item Slutt: 26/1
	\item Bemanning: Alle 
	\item Leveranse: Ingen leveranse 
	\item Beskrivelse: Opprette databasen som skal lagre dataene til værstasjonen.
    \item Prioritet:
	\end{compactitem}
    
    \item [Aktivitetet 5:] Mota sensorer
	\begin{compactitem}
	\item Start:20/1
	\item Slutt: 26/1
	\item Bemanning Alle
	\item Leveranse: Prosjektgruppe
	\item Beskrivelse: Venter på å mota sensorene fra leverandør.
    \item Prioritet:
	\end{compactitem}
    
    \item [Aktivitetet 6:] Tilpasse sensorene til Ethernut 2.1
	\begin{compactitem}
	\item Start: 26/1
	\item Slutt: 9/2
	\item Bemanning: Alle 
	\item Leveranse: Ingen leveranse
	\item Beskrivelse: Gjøre eventuelle endringer på sensorene for å kunne koble de til Ethernut 2.1.
    \item Prioritet:
	\end{compactitem}
    
    \item [Aktivitetet 7:] Programmere Ethernut 2.1
	\begin{compactitem}
	\item Start: 26/1
	\item Slutt: 16/2
	\item Bemanning: Alle 
	\item Leveranse: Ingen leveranse
	\item Beskrivelse: Sikre kommunikasjon mellom Ethernut 2.1 og sensorene, og mellom Ethernut 2.1 og databasen.  
    \item Prioritet:
	\end{compactitem}
    
    \item [Aktivitetet 8:] Første versjon av prosjektrapport
	\begin{compactitem}
	\item Start: 17/1
	\item Slutt: 14/3
	\item Bemanning: Alle 
	\item Leveranse: Einar Von Krogh 
	\item Beskrivelse: Sette opp et skjellett med overskrifter og tilhørende undertittler så langt det lar seg gjøre.
    \item Prioritet:
	\end{compactitem}
    
    \item [Aktivitetet 9:] Teste systemet på skolen
	\begin{compactitem}
	\item Start: 17/3
	\item Slutt: 24/3
	\item Bemanning: Alle
	\item Leveranse: Ingen leveranse
	\item Beskrivelse: Sette opp værstasjonen på skolen og se at alt fungere som det skal.
    \item Prioritet:
	\end{compactitem}
    
    \item [Aktivitetet 10:] Innstallere værstasjonen i Hessdalen
	\begin{compactitem}
	\item Start: 25/3
	\item Slutt: 5/4
	\item Bemanning: Alle
	\item Leveranse: Oppdragsgiver
	\item Beskrivelse: Systemet skal innstalleres i Hessdalen.
    \item Prioritet:
	\end{compactitem}
    
     \item [Aktivitetet 11:] Lage nettside for værstasjonen
	\begin{compactitem}
	\item Start: 6/4
	\item Slutt: 18/4
	\item Bemanning: Alle
	\item Leveranse: Ingen leveranse
	\item Beskrivelse: Lage en nettside for værstasjonen som presenterer dataene.
    \item Prioritet:
	\end{compactitem}
    
    \item [Aktivitetet 12:] Andre versjon av prosjektrapport
	\begin{compactitem}
	\item Start: 14/3
	\item Slutt: 25/4
	\item Bemanning: Alle
	\item Leveranse: Einar Von Krogh
	\item Beskrivelse: Det vi har fått gjort så langt skal være dokumentert i prosjektrapporten.
    \item Prioritet:
	\end{compactitem}
    
    \item [Aktivitetet 13:] Mediastrategi
	\begin{compactitem}
	\item Start: 25/4
	\item Slutt: 9/5
	\item Bemanning: Alle.
	\item Leveranse: Einar Von Krogh
	\item Beskrivelse: Lage en mediastrategi for prosjektet.
    \item Prioritet:
	\end{compactitem}
    
    \item [Aktivitetet 14:] Prosjektrapport med vedlegg pluss eventuelt ferdig produkt skal leveres
	\begin{compactitem}
	\item Start: 25/4
	\item Slutt: 23/5
	\item Bemanning: Alle
	\item Leveranse: Einar Von Krogh
	\item Beskrivelse: Alt skal være dokumentert og klart for levering.
    \item Prioritet:
	\end{compactitem}
    
    \item [Aktivitetet 15:] Opphenging av prosjektplakat
	\begin{compactitem}
	\item Start: 23/5
	\item Slutt: 2/6
	\item Bemanning: Alle
	\item Leveranse: Henges opp
	\item Beskrivelse: Lage en plakat som viser hva vi har gjort.
    \item Prioritet:
	\end{compactitem}
    
    \item [Aktivitetet 16:] Presentasjon av prosjektet
	\begin{compactitem}
	\item Start: 4/6
	\item Slutt: 6/6
	\item Bemanning: Alle
	\item Leveranse: Oppdragsgiver, veileder og sensor.
	\item Beskrivelse: Presentere prosjektet.
    \item Prioritet:
	\end{compactitem}
    

\end{compactdesc}


Det kan være lurt å lage en prioritert plan, dvs. at noen av oppgavene vil bli bli gjennomført hvis det blir tid og anledning til det. Det er vanskelig å planlegge et prosjekt, nettopp derfor kan det være lurt at planen sier at dette vil vi oppnå som et minimum, og så kommer et antall prioriterte oppgaver. Dere vil også antagelig få behov for å re-planlegge underveis.  Det er også fornuftig å gjøre en kort risikoanalyse av prosjekt, og peke på kritiske faktorer og eventuelle flaskehalser. Vår spesielt oppmerksomme på at det er betenkelig å gjøre prosjektet avhenging av utstyr, programvare eller liknende som ikke er tilgjengelig ved prosjektstart.

\section*{Gjennomføring}

Møter med arbeidsgiver vil i hovedsak foregå hver XX dag, og med veileder hver 14 dag. Siden både arbeidsgiver og veileder jobber på HiØ, vil det til nøds være mulig med uanmeldte besøk dersom vi har spørsmål. Vi vil hovedsaklig benytte mail for å kontakte arbeidsgiver og veileder. Ved disse møtene ønsker vi ærlige og konkrete tilbakemeldinger som gir oss en pekepinn på hva som evt. kan gjøres annerledes og hvor fornøyd arbeidsgiver/veileder er med det vi produserer.\\

Den eneste predefinerte rollen vi har bestemt oss for å ha i gruppen, er sekretær. Sekretæren vil i all hovedsak stå for skriving av møte- og ukesrapporter, men også få ansvaret for andre oppgaver som passer denne rollen. Sekretæransvaret vil ha ukentlig rundgang slik at alle både for prøvd seg på det, og for å unngå skjev fordeling i arbeidsmenge. Vi har altså valgt å ikke ha noen lederrolle, da denne gruppen har jobbet sammen tidligere, og av erfaring tar vi best slike avgjørelser i plenum. \\

Vi skal ikke bruke en spesifikk arbeidsteknikk som f.eks SCRUM, men vi vil fordele oppgaver på gruppemedlemmene og sette frister disse oppgavene skal være ferdigstilt til. Dette for å sikre at vi får en jevn fremgang i prosjektet.\\

For å overholde versjonskontrol og sikre at vi hele tiden har backups av prosjektet, vil vi bruke GitHub (https://github.com/) til lagring av ALLE filer. Her hver eneste endring, og man har mulighet til å hente opp tidligere versjoner av filer. Dette kombinert med å bruke LaTeX (http://www.latex-project.org/) som gir mye mer ryddige tekstfiler i ukompilert form vil gi oss god versjonskontroll.\\

Dersom et gruppemedlem ikke overholder frister for innlevering uten noen spesifikk grunn, vil vi bli nødt til å ta dette opp med personen og prøve å finne en løsning på problemet. Dersom et gruppemedlem pga. sykdom, o.l ikke kan overholde frister, fører dette til at vi må delegere mer arbeid på de andre, og muligens prioritere oppgavene slik at vi sikrer at det viktigste blir ferdigstilt.

\end{document}