\documentclass[11pt,a4paper]{report} 

\usepackage[utf8x]{inputenc} 
\usepackage[norsk]{babel} 
\usepackage{lipsum,paralist}
\usepackage[none]{hyphenat}
\usepackage[colorinlistoftodos]{todonotes}
\usepackage{float}
\usepackage[top=1in, bottom=1in, left=1in, right=1in]{geometry}

\begin{document}
\title{
Forprosjektrapport\\
\vspace{2cm}
Prosjektets tittel\\
Gruppe 15
}
\author{
\LARGE Mikael Johansen Grimstad\\
\LARGE Kristian Norum Karlsen\\
\LARGE Kristoffer Jensen\\
\LARGE Morten Lindstad
}
\maketitle

\section*{Prosjektgruppen}
Mikael Johansen Grimstad(født 23.08.1991) gikk tidligere studiespesialisering med realfag ved Frederik II VGS(2007-2010). I året som fulgte utførte han verneplikt som gardist ved Hans Majestet Kongens Garde. Videre valgte å studere Informatikk - Design og utvikling av IT-systemer ved Høgskolen i Østfold fordi han alltid har hatt en stor interesse for elektronikk, data og programmering.\\

Kristian Norum Karlsen(født 27.09.1991) gikk tidligere studiespesialisering med realfag ved Frederik II VGS(2007-2010). I året som fulgte utførte han verneplikt som sambandsmann i Sambandsbataljonen. Videre valgte han bachelorstudiet dataingeniør ved Høgskolen i Østfold da han var der på elevbesøk i 2. klasse på videregående, og dette ble et naturlig valg med tanke på hans interesser.\\

Kristoffer Jensen(født 14.11.1991) gikk tidligere studiespesialiserende elektrofag ved Glemmen VGS(2007-2010). I året som fulgte utførte han verneplikt som luftvernartillerist ved Ørland hovedflystasjon. Videre valgte han bachelorstudiet dataingeniør ved Høgskolen i Østfold da dette virket som en naturlig valg med tanke på hans interesser.\\

Morten Lindstad(født 10.05.1992) gikk tidligere på elektrofag med studiekompetanse(2008-2011) ved Glemmen VGS før han startet på dataningeniørlinja på Høgskolen i Østfold. Dette viste seg å ha flere fordeler med tanke på at flere av faga går igjen med hva han hadde på videregående. Han har alltid hatt en stor interesse for elektronikk og data, alt fra elektriske gitarer til dataspill. Dette er også en fordel fordi han og vil mest sannsynlig ha en enklere forståelse for bacheloroppgaven.\\

Det var ikke noe tilfeldighet at det var akkurat vi fire som havnet på samme gruppe. Vi har jobbet som en gruppe i flere annledninger tidligere. Vi har samarbeidet i fag som Datakommunikasjon, Bildebehandling og mønstergjenkjenning og Integrerte IT-systemer. 

\section*{Oppdragsgiver}

Oppdragsgiveren for vårt bachelorprosjekt er Erling P. Strand. Han er høgskolelektor ved Høgskolen i Østfold. Han underviser i fag som tressfysikk, datakommunikasjon, fysikk/kjemi og datateknikk. Han har tidligere bakgrunn som sivilingeniør og utdannet seg til dette i Trondheim.\\

Han har i tillegg et prosjekt ved siden av han kaller Prosjekt Hessdalen. Etter det ble oppdaget ukjente lysfenomener i Hessdalen i slutten av 1981 ble Prosjekt Hessdalen grunnlagt. Det har i senere tid blitt mye mediaoppmerksomhet rundt "Hessdalsfenomenet". Strand er i dag leder av dette prosjektet. \\

Erling Strand vil hente mest mulig informasjon om disse lysfenomenene ved hjelp av kameraer og sensorer for å registere bilder og bevegelse. Han ville ha en studentgruppe fra HiØ som skal sette opp en komplett værstasjon bestående av diverse sensorer i Hessdalen.

\section*{Oppdraget}

Gruppen skal konstruere to værstasjoner. Disse skal monteres i Hessdalen. Disse værstasjonene skal bestå av en rekke sensorer som er koblet til mikroprosessoren Ethernut 2.1. Mikroprosessorene skal programmeres slik at de lagrer data i en database på HiØ.\\

Disse dataene skal presenteres på nettsiden til Prosjekt Hessdalen. Ved hjelp av disse dataene kan oppdragsgiver undersøke om værforhold er en innvirkning på lysfenomenene i Hessdalen. Værstasjonene skal måle temperatur, lufttrykk, vindhastighet, vindretning og luftfuktighet.
\subsection*{Formål}

\begin{compactitem}
\item [{\bf Hovedmål}] Måle værdata i Hessdalen og presentere dette på en nettside.
\begin{compactitem}
\item [{\bf  Delmål 1}] Konstruere to værstasjoner.
\item [{\bf  Delmål 2}] Programmere mikroprosessorene.
\item [{\bf  Delmål 3}] Lage nettside.
\end{compactitem}
\end{compactitem}

\subsection*{Leveranser}

Gruppen skal levere to værstasjoner som skal installeres i Hessdalen. Disse skal henges opp i trær, og skal kobles til Prosjekt Hessdalens hovedstasjon.\\
Det skal utvikles et program som skal ta seg av midlertidig lagring av værdata i Hessdalen, før dette sendes videre til HiØ for permanent lagring. Dette skal så presenteres grafisk på en nettside.

\subsection*{Metode}

Gruppen skal først lese seg opp på den teorien som er tilgjengelig. Før vi går over til det praktiske. Gruppen sitter på en del kunnskap fra tidligere fag som vil komme til nytte.
Gruppen kommer til å arbeide ved å benytte en induktiv metode. Det vil si at vi skal lære mens vi arbeider.

\section*{Prosjektplan}

\begin{compactdesc}

\item [Aktivitetet 1:] Hjemmeside
	\begin{compactitem}
	\item Start: 1/1
	\item Slutt: 10/1
	\item Bemanning: Alle
	\item Leveranse: Ingen leveranse 
	\item Beskrivelse: Her skal det legges ut en kort beskrivelse av prosjektet, alt av dokumentasjon tilhørende 
    	  prosjektet og kontaktinformasjon til gruppemedlemmene.
	\end{compactitem}
    
	\item [Aktivitetet 2:] Forprosjektrapport
	\begin{compactitem}
	\item Start: 10/1
	\item Slutt: 17/1
	\item Bemanning Alle
	\item Leveranse: Einar Von Krogh
	\item Beskrivelse: Beskrivelse av prosjektgruppen, oppdragsgiver, oppdraget, formål, leveranser, metode, prosjektplan og gjennomføring.
	\end{compactitem}
    
    \item [Aktivitetet 3:] Finne sensorer
	\begin{compactitem}
	\item Start: 6/1
	\item Slutt: 19/1
	\item Bemanning: Alle
	\item Leveranse: Oppdragsgiver
	\item Beskrivelse: Finne de ulike sensorene som passer til Ethernut 2.1
	\end{compactitem}
    
    \item [Aktivitetet 4:] Opprette databasen
	\begin{compactitem}
	\item Start: 20/1
	\item Slutt: 26/1
	\item Bemanning: Alle
	\item Leveranse: Ingen leveranse 
	\item Beskrivelse: Opprette databasen som skal lagre dataene til værstasjonen.
	\end{compactitem}
    
    \item [Aktivitetet 5:] Motta sensorer
	\begin{compactitem}
	\item Start:20/1
	\item Slutt: 26/1
	\item Bemanning Alle
	\item Leveranse: Prosjektgruppe
	\item Beskrivelse: Venter på å mota sensorene fra leverandør.
	\end{compactitem}
    
    \item [Aktivitetet 6:] Tilpasse sensorene til Ethernut 2.1
	\begin{compactitem}
	\item Start: 26/1
	\item Slutt: 9/2
	\item Bemanning: Alle
	\item Leveranse: Ingen leveranse
	\item Beskrivelse: Gjøre eventuelle endringer på sensorene for å kunne koble de til Ethernut 2.1.
	\end{compactitem}
    
    \item [Aktivitetet 7:] Programmere Ethernut 2.1
	\begin{compactitem}
	\item Start: 26/1
	\item Slutt: 16/2
	\item Bemanning: Alle
	\item Leveranse: Ingen leveranse
	\item Beskrivelse: Sikre kommunikasjon mellom Ethernut 2.1 og sensorene, og mellom Ethernut 2.1 og databasen.  
	\end{compactitem}
    
    \item [Aktivitetet 8:] Første versjon av prosjektrapport
	\begin{compactitem}
	\item Start: 17/1
	\item Slutt: 14/3
	\item Bemanning: Alle
	\item Leveranse: Einar Von Krogh
	\item Beskrivelse: Sette opp et skjelett med overskrifter og tilhørende undertittler så langt det lar seg gjøre.
	\end{compactitem}
    
    \item [Aktivitetet 9:] Teste systemet på skolen
	\begin{compactitem}
	\item Start: 17/3
	\item Slutt: 24/3
	\item Bemanning: Alle
	\item Leveranse: Ingen leveranse
	\item Beskrivelse: Sette opp værstasjonen på skolen og se at alt fungere som det skal.
	\end{compactitem}
    
    \item [Aktivitetet 10:] Installere værstasjonen i Hessdalen
	\begin{compactitem}
	\item Start: 25/3
	\item Slutt: 5/4
	\item Bemanning: Alle
	\item Leveranse: Oppdragsgiver
	\item Beskrivelse: Systemet skal installeres i Hessdalen.
	\end{compactitem}
    
     \item [Aktivitetet 11:] Lage nettside for værstasjonen
	\begin{compactitem}
	\item Start: 6/4
	\item Slutt: 18/4
	\item Bemanning: Alle
	\item Leveranse: Ingen leveranse
	\item Beskrivelse: Lage en nettside for værstasjonen som presenterer dataene.
	\end{compactitem}
    
    \item [Aktivitetet 12:] Andre versjon av prosjektrapport
	\begin{compactitem}
	\item Start: 14/3
	\item Slutt: 25/4
	\item Bemanning: Alle
	\item Leveranse: Einar Von Krogh
	\item Beskrivelse: Det vi har fått gjort så langt skal være dokumentert i prosjektrapporten.
	\end{compactitem}
    
    \item [Aktivitetet 13:] Mediestrategi
	\begin{compactitem}
	\item Start: 25/4
	\item Slutt: 9/5
	\item Bemanning: Alle.
	\item Leveranse: Einar Von Krogh
	\item Beskrivelse: Lage en mediestrategi for prosjektet.
	\end{compactitem}
    
    \item [Aktivitetet 14:] Prosjektrapport med vedlegg pluss eventuelt ferdig produkt skal leveres
	\begin{compactitem}
	\item Start: 25/4
	\item Slutt: 23/5
	\item Bemanning: Alle
	\item Leveranse: Einar Von Krogh
	\item Beskrivelse: Alt skal være dokumentert og klart for levering.
	\end{compactitem}
    
    \item [Aktivitetet 15:] Opphenging av prosjektplakat
	\begin{compactitem}
	\item Start: 23/5 
	\item Slutt: 2/6
	\item Bemanning: Alle
	\item Leveranse: Henges opp
	\item Beskrivelse: Lage en plakat som viser hva vi har gjort.
	\end{compactitem}
    
    \item [Aktivitetet 16:] Presentasjon av prosjektet
	\begin{compactitem}
	\item Start: 4/6
	\item Slutt: 6/6
	\item Bemanning: Alle
	\item Leveranse: Oppdragsgiver, veileder og sensor.
	\item Beskrivelse: Presentere prosjektet.
	\end{compactitem}
    
\end{compactdesc}

\begin{figure}[H]
\centering
\includegraphics[width=0.66\textwidth]{gantUtkast.png}
\caption{\label{fig:frog}Gantt-skjema av prosjektplan}
\end{figure}

\section*{Gjennomføring}

Møter med arbeidsgiver vil i hovedsak foregå ukentlig, og med veileder hver 14. dag. Siden både arbeidsgiver og veileder jobber på HiØ, vil det til nøds være mulig med uanmeldte besøk dersom vi har spørsmål. Vi vil hovedsaklig benytte mail for å kontakte arbeidsgiver og veileder. Ved disse møtene ønsker vi ærlige og konkrete tilbakemeldinger som gir oss en pekepinn på hva som evt. kan gjøres annerledes og hvor fornøyd arbeidsgiver/veileder er med det vi produserer.\\

Den eneste predefinerte rollen vi har bestemt oss for å ha i gruppen, er sekretær. Sekretæren vil i all hovedsak stå for skriving av møte- og ukesrapporter, men også få ansvaret for andre oppgaver som passer denne rollen. Sekretæransvaret vil ha ukentlig rundgang slik at alle både for prøvd seg på det, og for å unngå skjev fordeling i arbeidsmenge. Vi har altså valgt å ikke ha noen lederrolle, da denne gruppen har jobbet sammen tidligere, og av erfaring tar vi best slike avgjørelser i plenum. \\

Vi skal ikke bruke en spesifikk arbeidsteknikk som f.eks SCRUM, men vi vil fordele oppgaver på gruppemedlemmene og sette frister disse oppgavene skal være ferdigstilt til. Dette for å sikre at vi får en jevn fremgang i prosjektet.\\

For å overholde versjonskontrol og sikre at vi hele tiden har backups av prosjektet, vil vi bruke GitHub (https://github.com/) til lagring av ALLE filer. Her hver eneste endring, og man har mulighet til å hente opp tidligere versjoner av filer. Dette kombinert med å bruke LaTeX (http://www.latex-project.org/) som gir mer ryddige tekstfiler i ukompilert form vil gi oss god versjonskontroll.\\

Dersom et gruppemedlem ikke overholder frister for innlevering uten noen spesifikk grunn, vil vi bli nødt til å ta dette opp med personen og prøve å finne en løsning på problemet. Dersom et gruppemedlem pga. sykdom, o.l ikke kan overholde frister, fører dette til at vi må delegere mer arbeid på de andre, og muligens prioritere oppgavene slik at vi sikrer at det viktigste blir ferdigstilt.

\end{document}