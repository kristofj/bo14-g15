%\cleardoublepage
\chapter{Testing}
\label{chap:evaluation} 


I dette kapittelet vil vi ta for oss hvordan vi har testet produktet. Forklare hvordan de ulike delene av produktet har blitt testet og hvem som har testet det. Forklare kort om hva vi var ute etter med testen og om produktet holdt mål.


\section{Testing av sensorene}

Her kan vi skrive litt om hvordan vi testet de forskjellige sensorene.
Kanskje undertittel(subsection) til hver måler?

\subsection{Testing av temperaturmåler}

Tekst


\section{Testing av systemet i Hessdalen}

Mandag 5. mai dro bachelorgruppen med Erling Strand opp til Hessdalen. Det var planlagt lenge at vi skulle dra denne datoen og bruke denne uka til å montere værstasjonene i Hessdalen, hvis vi ble ferdige i tide.

Turen oppover tok rundt 8 timer å kjøre med et par småstopp. Erling Strand hadde fått leid oss et hus vi skulle bo i under oppholdet. Mandags kveld ble brukt til den siste testingen av systemet og rette opp i eventuelle feil i koden som kunne få systemet til å krasje.

\begin{figure}[H]
  \centering
  \includegraphics[width=0.5\textwidth]{sisteTest1}
  \caption{Den aller siste testingen av værstasjonen}
\end{figure}

\begin{figure}[H]
  \centering
  \includegraphics[width=0.5\textwidth]{sisteTest2}
  \caption{Den aller siste testingen av værstasjonen}
\end{figure}

Tirsdags morgen dro vi opp til Bluebox for å montere den første værstasjonen. Dette var omtrent 10 minutter fra der vi bodde\\

Imens vi koblet opp den første værstasjonen kom NRK og lagde en repotasje om hessdalsfenomenet. Det var "Normal Galskap" med Are Sende Osen som programleder. Dette blir da sendt i løpet av høsten 2014.\\

Etter vi var ferdige med å sette opp den første værstasjonen, gikk vi til det andre måletstedet som var et par hundre meter lengre opp i dalen der vi begynte å sette opp den andre værstasjonen. Vi brukte onsdag formiddag til å bli helt ferdige med å sette opp den andre værstasjonen.

