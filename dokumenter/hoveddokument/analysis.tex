%\cleardoublepage
\chapter{Analyse (Generisk tittel)}
\label{chap:analysis}
\meta{
Kapittelet tar for seg analysedelen av arbeidet. Den består av to hoveddeler, en grundig beskrivelse av oppgaven basert på skissen gitt av oppdragsgiver, og en undersøkelse av hva som finnes av relatert arbeid, {\em best practise} og relevant teknologi. 
}

\section{Analyse}
\subsection{Grundig beskrivelse av oppgaven basert p� skissen gitt av oppdragsgiver (trenger ny tittel)}
Oppdragsgiver �nsker to v�rstasjoner for jevnlig innhenting og rapportering av v�ret i Hessdalen. Disse v�rstasjonene skal bygges og settes opp i forbindelse med "Project Hessdalen".  

\subsection{Project Hessdalen}
I 1981 ble det observert flere lys flytende over himmelen i en bygd ved navn Hessdalen "Project Hessdalen" er et prosjekt startet i 1983 for � finne ut mer om de uforklarte lysfenomenene som regelmessig blir observert i Hessdalen (Holt�len kommune, S�r-Tr�ndelag).

\subsection{Hva finnes av relatert arbeid?}

Det har vært mange tidligere prosjekter i forbindelse med Hessdalen, men ingen prosjekter der værdata skal representeres.
I dag kan nesten hvem som helst kjøpe et værstasjon-system og koble opp, og programmere værstasjonen på sin måte. 
Det finnes systemer som er mer brukervennlig enn andre, som ikke er så erfarne på området.

%\avsnitt
\\

Det er mange som har laget sin egen værstasjson, en av de kaller seg "Dr Bunson" (kilde: http://www.drbunsen.org/building-a-weather-station/).
Han lager en liknende værstasjon som det vi har i oppgave i å lage.


\subsection{Kan vi lære noe av disse prosjektene?}

Vi kan ha mye å lære av og jobbe med et slikt prosjekt. Det er mye ny informasjon og nye fagområder for flere av gruppemedlemene



\subsubsection{subb subb section}


