%\cleardoublepage
\chapter{Analyse}
\label{chap:analysis}
\subsection{Grundig beskrivelse av oppgaven basert p� skissen gitt av oppdragsgiver (trenger ny tittel)}
Oppdragsgiver �nsker to v�rstasjoner for jevnlig innhenting og rapportering av v�ret i Hessdalen. Disse v�rstasjonene skal bygges og settes opp i forbindelse med "Project Hessdalen".  

\subsection{Project Hessdalen}
I 1981 ble det observert flere lys flytende over himmelen i en bygd ved navn Hessdalen "Project Hessdalen" er et prosjekt startet i 1983 for � finne ut mer om de uforklarte lysfenomenene som regelmessig blir observert i Hessdalen (Holt�len kommune, S�r-Tr�ndelag).

\subsection{Hva finnes av relatert arbeid?}

Det har vært mange tidligere prosjekter i forbindelse med Hessdalen, men ingen prosjekter der værdata skal representeres. I dag kan nesten hvem som helst kjøpe et værstasjon-system å koble opp, og programmere værstasjonen på sin måte. Det finnes systemer som er mer brukervennlig enn andre, hvis man ikke er erfarne på det området. \\

\subsubsection{Dr. Bunson\cite{nettside:drbunson}}
Det er mange eksempler på andre som har laget sin egen værstasjson. En av de  er Seth Brown og kaller seg "Dr Bunson". Han har laget en liknende værstasjon som det vi skal. Bakgrunnen for at han bygde sin værstasjon er at han bor på et område det der er dårlige representative værdata, så han valgte å ta saken i egne hender. \\ \\
Han vurderte først å lage sin egen værstasjon fra bunnen, basert på enten Arduino\cite{nettside:arduino} eller Tessel\cite{nettside:tessel}, men fant ut at ingen av disse ville tåle påkjenningene av ekstremværet der han bor (Mount Washington). Derfor bestemte han seg for å kjøpe Davis 6250 Vantage Vue\cite{nettside:davis6250}, en ferdigbygd kombinert vind-, regn- og temperaturmåler. Denne værstasjonen sender data trådløst, og serveren som tar i mot dette er en Raspberry Pi model B\cite{nettside:rpi} med en Edimax EW-7811Un trådløs nettverksadapter\cite{nettside:edimax}. \\ \\
Serveren programmerte han i Node\cite{nettside:nodejs}, et JavaScript-bibliotek for nettverksservere. Han brukte også JavaScript for å få visualisert data på skjermer i huset. Han har også programmert et varslingssystem på serversiden ved bruk av Nodemailer\cite{nettside:nodemailer} som sender SMS-varslinger til telefonen hans når været når ekstreme verdier.
\begin{figure}[H]
  \centering
  \includegraphics[width=0.5\textwidth]{drbunson}
  \caption{Skjermdump av Dr. Bunsons JavaScript-presentasjon av værdata.}
\end{figure}

\subsubsection{Arduino's weather station\cite{nettside:instructables}}
En annen som har laget sin egen værstasjon skrev en artikkel på hvordan han satt opp hele systemet, steg for steg. Han oppgir ikke navn på nettsiden men han går under brukernavnet bram2202. Han bruker sensorer for trykk, temperatur, luftfuktighet og bevegelse. Han viser og skriver detaljert hva han bruker av utstyr, kretsskjemaer og hvordan han koblet alt mot en Arduino mikrokontroller\cite{nettside:arduino} og han har tatt med en sender og en mottaker mellom mikrokontrolleren og sensorene.\\ \\
Værstasjonen ble bygd på en slik måte at han alltid kunne ha plass til utvidelser. Han viser ved et eksempel at han at koblet opp og programmerte en sensor som måler nedbør. Alt han trengte var en måler for nedbør og en boks for å sikre denne måleren mot vann. Måleren kan bare kobles på en ledig port på mikrokontrolleren.
\begin{figure}[H]
  \centering
  \includegraphics[width=0.5\textwidth]{arduinos_vaerstasjon}
  \caption{Den ferdige værstasjonen til bram2202 som viser temperatur, luftfuktighet og trykk}
\end{figure}

\subsection{Kan vi lære noe av disse prosjektene?}
Dr. Bunson mente at mikrokontrollerkort ikke tålte nok til å brukes som værstasjoner der han bodde, men det var hovedsaklig på grunn av ekstrem vind. I Hessdalen er det meste kulde som kan bli problemet, da temperaturen kan krype under 30 minusgrader\cite{nettside:yr150313}. Siden det er påkrevd i oppgaven at vi bruker Ethernut 2.1, er det eneste vi kan gjøre å beskytte kortet godt fra elementene. Med denne informasjonen i bakhodet, vil en prioritet for oss være å bygge vår værstasjon inn i en godt beskyttende boks. Grensesnittet Dr. Bunson har brukt for fremvisning av data kan også være en inspirasjon når vi skal utforme vår side for datapresentasjon.\\ \\
Bram2202 forklarer ikke koden han har skrevet for prosjektet, men han har vedlagt det som en link som man kan laste ned med hans instillinger for værstasjonen hvis man ville lagd en helt liknende værstasjon. Det er kun kretsskjemaer og oppkoblingen av målerne han har forklart. Det vi kan bruke av dette prosjektet er at vi kan lage en liknende boks, der vi beskytter alle kretsene for vann. I dette prosjektet blir dataene representert på et LED-display. Vi skal isteden lagre dataene i en database som vi kan hente opp, det vil si all data som har blitt regisrert fra første gang værstasjonen er oppe og går til nåtid.

\section{Værstasjonens historie}

\subsection{Fra observasjoner til datamodeller}
Mennesket har alltid vært helt avhengig av været, og behovet for å forutsi værgudenes neste trekk har eksistert til alle tider. 
Været har alltid spilt en stor rolle for blant annet landbruk, skipsfart og fiske. Derfor har mennesket i årtusener hatt et intenst ønske om å kunne forutsi værgudenes neste trekk.
I begynnelsen bygde forutsigelsene bare på observasjoner og tilfeldige målinger, men på 1700-tallet utviklet vitenskapsfolk instrumentene som la grunnlaget for at den moderne meteorologien kunne bli en reell vitenskap. Dermed kunne forskere forstå de faktorene som bestemmer værets utvikling, og lage nyttige værmeldinger, som for eksempel stormvarsler.\\
\begin{figure}[H]
  \centering
  \includegraphics[width=0.5\textwidth]{himmel}
  \caption{Uvær}
\end{figure}

\subsection{Måling av lufttrykk}
Ingen visste at luft har tyngde\cite{nettside:historienet}, før italieneren Evangelista Torricelli fant opp kvikksølvbarometeret i 1643. Han la merke til at en kvikksølvsøyle sto høyere i godt vær enn i blåst og regn. I 1714 fant Gabriel Fahrenheit ut at det gikk an å måle temperaturen i luft og vann med et kvikksølvtermometer. Det teoretiske grunnlaget ble også lagt på 1700-tallet, da sveitseren Daniel Bernoulli ga ut verket «Hydrodynamica».\\

\subsection{Første ballongferd}
I 1783 gjennomførte Jacques Alexandre Charles den første flygningen med en ballong fylt med hydrogen. Oppfinnelsen viste seg å være svært velegnet til å samle inn meteorologiske data fra atmosfæren. Men fordi ballonger var prisgitt værgudenes luner, sendte de første ballongskipperne opp små prøveballonger før de lettet selv.\\

\subsection{Utveksling av værdata}
Admiral Francis Beaufort foreslo i 1806 å dele vindstyrken i en skala på 12 trinn – den såkalte beaufortskalaen. Dette ga meteorologene en felles referanse. Da telegrafen ble oppfunnet senere på 1800-tallet, ble det for første gang praktisk mulig å innhente værdata fra større områder, og eksempelvis sende ut stormvarsler. I 1849 opprettet Smithsonian Institution et meteorologisk nettverk på tvers av USA, og lignende samarbeidsprosjekter mellom meteorologer ble etablert i Europa de neste 50 årene.\\

\subsection{Beregning av morgendagens værdata}
I 1922 fremla briten Lewis Fry Richardson i verket «Weather Prediction by Numerical Process» en rekke formler som gjorde det mulig å beregne morgendagens vær noenlunde sikkert. Problemet var bare at det tok 24 timer å foreta alle de kompliserte utregningene. Dermed var morgendagen over før værmeldingen var klar. Richardson planla å samle 64 000 meteorologer på ett sted, slik at de kunne løse hver sin del av regnestykket, men ideen ble aldri noe av.\\

\subsection{Radiosonden et stort framskritt}
Meteorologene fikk adgang til enda bedre data da to vitenskapsmenn i 1927 sendte meteorologiske måleinstrumenter opp i en ballong utstyrt med en radiosender. Målingene ble dermed sendt trådløst hjem, og verdens første radiosonde var en realitet. Ti år senere opprettet USAs meteorologiske institutt en fast værtjeneste basert på radiosonder i ballonger. Under andre verdenskrig ble radioteknikken utviklet videre, og midt i 1950-årene gjorde bedre materialer det mulig å sende opp gigantiske ballonger. Radiosondene kunne nå komme opp i 50 kilometers høyde og samle inn svært nøyaktige data.\\
\begin{figure}[H]
  \centering
  \includegraphics[width=0.5\textwidth]{radiosonde}
  \caption{Radiosonde}
\end{figure}

\subsection{Første værsateslitt}
Da de første datamaskinene ble bygd på 1950-tallet, kunne meteorologene beregne værets utvikling et helt døgn frem. Men ettersom de tidligste «elektronhjernene» bare hadde en brøkdel av kapasiteten til moderne datamaskiner, tok beregningene fremdeles nesten 24 timer. Nettopp fordi meteorologiske beregninger er så kompliserte, egner de seg for databehandling, og meteorologien var derfor med på å videreutvikle datateknologien.\\
Fra 1960-årene brakte værsatellittene enda flere data. Den første, Vanguard 2, ble sendt opp av den amerikanske marinen i 1959. Den skulle måle skydekkets tykkelse, men kom ut av kurs. Det gikk bedre med den neste værsatellitten, TIROS-1, sendt opp av NASA i 1960. Nå kunne all datainnsamling foregå automatisk.\\
\begin{figure}[H]
  \centering
  \includegraphics[width=0.5\textwidth]{satellit}
  \caption{Første vellykkede værsatellitten}
\end{figure}

\section{Hvordan det blir målt}

\subsection{Måling av lufttrykk}
De meteorologiske målingene tar sikte på å finne den romlige fordelingen av det hydrostatiske trykk\cite{nettside:snl}, som igjen bestemmer den storstilte luftsirkulasjonen. Lufttrykk angis i enheten pascal (Pa), men innen meteorologien nyttes enheten hektopascal (hPa) som gir tallverdi lik den tidligere nyttede enheten millibar (mb). Til mange tekniske formål angis trykket i atmosfærer eller mm kvikksølv (Hg). Én atmosfære (atm) gir uttrykk for tilnærmet normalt lufttrykk ved havnivå; 1 atm = 1013,25 hPa = 760 mm Hg. 1000 hPa = 750 mm Hg. Apparater for måling av lufttrykk kan være for eksempel et barometer.\\

\subsection{Måling av vind}

Den øyeblikkelige vind vil alltid variere\cite{nettside:snl}. I forbindelse med meteorologiske observasjoner og varsler gis, etter internasjonal overenskomst, vindens middelverdi for 10 minutter, som også ligger til grunn ved bruk av Beauforts vindskala.\\
Vindens fart øker med høyden. Nede på selve bakken er farten null, mens økningen skjer raskt de første metrene. Videre oppover øker farten, men ikke like raskt. Det er internasjonalt bestemt at vindmålinger for værvarslings- og klimaformål skal gjøres 10 meter over bakken. I noen hundre meters høyde dreier vindretningen etter hvert til høyre (på den nordlige halvkule), men i nærheten av åser, fjell og fjellkjeder er vinden i stor grad styrt av terrenget og høyde-endringen kan arte seg annerledes.\\

\subsubsection{Vindfarten}

Vindfarten oppgis i m/s, knop eller km/h. Begrepet vindhastighet omfatter i streng forstand både fart og retning, men brukes ofte bare om fart.\\
Både i forbindelse med observasjoner og varsler har en internasjonal vindskala stor utbredelse. Den bygger på Beauforts vindskala, som opprinnelig ble tatt i bruk først på 1800-tallet av den britiske admiral Francis Beaufort (1774–1857). Skalaen bygde opprinnelig på vindens virkning på seilfartøyer og er senere tilpasset instrumentmålinger av vindhastigheten. Vind angitt i skalaintervaller benevnes gjerne vindstyrke.\\


\subsubsection{Vindskalaen}

\begin{center}
 \begin{tabular}{ | p{2cm} | p{2cm} | p{2cm} | p{2cm} | p{2cm} | p{2cm} | }
 \hline
     Beauforts skala & Navn på vindstyrken & Vindstyrke i knop & Vindstyrke i m/s & Virkning på land & Virkning på sjøen \\ \hline
     0 & Stille & Mindre enn 1 & 0,0 - 0,2 & Røyken stiger rett opp & Sjøen er speilblank \\ \hline
     1 & Flau vind & 1 - 3 & 0,3 - 1,5 & Vindretningen sees av røykens drift & Krusninger danner seg på havflater \\ \hline
     2 & Svak vind & 4 - 6 & 1,6 - 3,3 & Følbar, rører på trærnes blader, løfter en vimpel & Små korte, men tydlige bølger med glatte kammer, som ikke brekker \\ \hline
     3 & Lett bris & 7 - 10 & 3,4 - 5,4 & Løv og småkvister rører seg, vinden strekker lette flagg og vimpler & Småbølgene begynner å toppe seg. Det dannes skum, som ser ut som glass. En og annen skumskavl kan forekomme \\ \hline
     4 & Laber bris & 11 - 16 & 5,5 - 7,9 & Løfter støv og løse papirer, rører på kvister og smågrener, strekker større flagg og vimpler & Bølgene blir lengre \\ \hline
     5 & Frisk bris & 17 - 21 & 8,0 - 10,7 & Småtrær med løv begynner å svaie, på vann begynner småbølgene å toppe seg & Middelstore bølger som har en mer utpreget langstrakt form og med mange skumskavler. Sjøsprøyt fra toppene kan forekomme \\ \hline
     6 & Liten kuling & 22 - 27 & 10,8 - 13,8 & Store grener og mindre stammer rører seg, det hviner i telefontrådene, det er vanskelig å bruke paraply, man merker motstand når man går & Store bølger begynner å dannes. Skumskavlene er større overalt. Gjerne noe sjøsprøyt \\ \hline
     7 & Stiv kuling & 28 - 33 & 13,9 - 17,1 & Hele trær rører seg, tungvint å gå mot vinden & Sjøen hoper seg opp, og hvitt skum fra bølgetopper som brekker, begynner å blåse i strimer i vindreningen \\ \hline
     8 & Sterk kuling & 34 - 40 & 17,2 - 20,7 & Brekker kvister av trærne, tungt å gå mot vinden & Middels høye bølger av større lengde. Bølgekammene begynner å brytes opp til sjørokk, som driver i tydelig markerte strimer med vinden \\ \hline
     9 & Liten storm & 41 - 47 & 20,8 - 24,4 & Hele store trær svaier og hiver, takstein kan blåse ned & Høye bølger. Tette skumstrimer driver i vindretningen. Sjørokket kan minske synsvidden \\ \hline
     10 & Full storm & 48 - 55 & 24,5 - 28,4 & Sjelden inne i landet, trær rykkes opp med rot, store skader på hus & Meget høye bølger med lange overhengende kammer. Skummet, som dannes i store flak, driver med vinden i tette hvite strimer, så sjøen får et hvitaktig utseende. Synsvidden nedsettes \\ \hline
     11 & Sterk storm & 56 - 63 & 28,5 - 32,6 & Meget sjeldent, følges av store ødeleggelser & Ualminnelig høye bølger. Sjøen er fullstendig dekket av lange, hvite skumflak som ligger i vindens retning. Overalt blåser bølgekammene til frådelignende skum. Sjørokket nedsetter synsvidden \\ \hline
     12 & Orkan & Over 63 & Over 32,6 & Sjeldent. Store ødeleggelser & Luften er fylt av skum og sjørokk som nedsetter synsvidden betydelig. Sjøen er fullstendig hvit av drivende skum \\ \hline
\end{tabular}
\end{center}

\subsection{Vindretning}

De to mest praktiske måtene å måle vindretningen på er med en vindpølse eller en vindfløy. \\

\subsubsection{Vindfløy}

Vindfløy er et enkelt instrument som viser vindretningen. Vanligvis er en vindfløy utført som en usymmetrisk figur som kan dreie seg om en vertikal akse. Figurens tyngdepunkt er normalt sammenfallende med omdreiningsaksen. Derved vil ikke vindføyen påvirkes om aksen avviker noe fra vertikal. Figurens areal er imidlertid utført slik at det er større på den ene siden, den «tyngste» siden. Dermed vil denne siden svinge fra vinden. Ofte er det til fløystangen festet et kors eller en sirkel med angivelse av himmelretninger. En moderne form for vindføy, Vindex, formet som en (liten) pil med (stor) styrefjær, brukes på seilfartøyer\cite{nettside:wikipedia}\\

\begin{figure}[H]
  \centering
  \includegraphics[width=0.5\textwidth]{vindfloy}
  \caption{Vindfløy i smijern på industribygning i Zeitz, Tyskland }
\end{figure}

\subsubsection{Vindpølse}

Vindpølse er en stor konisk tekstilsekk, åpen i begge ender, som viser vindretning og vindhastighet. Vindpølser er mest brukt på flyplasser, helikopterplasser og på bruer.\cite{nettside:wikipedia}\\

\begin{figure}[H]
  \centering
  \includegraphics[width=0.5\textwidth]{vindpolse}
  \caption{Vindpølse}
\end{figure}

Vindpølser skal generelt tilfredsstille følgende krav:
\begin{itemize}
\item    den skal gi presis indikasjon på vindretning og grov indikasjon på vindstyrke.
\item    den skal være traktformet og åpen i begge ender.
\item    den skal være oransje eller oransje og hvit. Dersom oransje og hvit velges, skal fargene være i form av 5 sirkulære bånd der første og siste bånd er oransje.
\end{itemize}
I tillegg for flyplasser:
\begin{itemize}
\item    lengden skal være minst 3,6 meter.
\item    den største åpningen skal ikke være mindre enn 0,9 m, og den minste åpningen ikke mindre enn 0,3 m.
\end{itemize}
I tillegg for helikopterplasser:
\begin{itemize}
\item    lengden skal være minst 2,4 meter,
\item    den største åpningen skal ikke være mindre enn 0,6 m, og den minste åpningen ikke mindre enn 0,3 m,
\end{itemize}


