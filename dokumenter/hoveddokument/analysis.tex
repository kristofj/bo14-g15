%\cleardoublepage
\chapter{Analyse}
\label{chap:analysis}
\subsection{Grundig beskrivelse av oppgaven basert p� skissen gitt av oppdragsgiver (trenger ny tittel)}
Oppdragsgiver �nsker to v�rstasjoner for jevnlig innhenting og rapportering av v�ret i Hessdalen. Disse v�rstasjonene skal bygges og settes opp i forbindelse med "Project Hessdalen".  

\subsection{Project Hessdalen}
I 1981 ble det observert flere lys flytende over himmelen i en bygd ved navn Hessdalen "Project Hessdalen" er et prosjekt startet i 1983 for � finne ut mer om de uforklarte lysfenomenene som regelmessig blir observert i Hessdalen (Holt�len kommune, S�r-Tr�ndelag).

\subsection{Relatert arbeid}

Det har vært mange tidligere prosjekter i forbindelse med Hessdalen, men ingen prosjekter der værdata skal representeres. I dag kan nesten hvem som helst kjøpe en egen værstasjon som et eget system. Det finnes systemer som er mer brukervennlig enn andre, hvis man ikke er erfaren på det området.

\subsubsection{Dr. Bunson\cite{nettside:drbunson}}
Det er mange eksempler på andre som har laget sin egen værstasjson. En av de er Dr. Seth Brown(på bloggen sin bruker han navnet Dr. Bunson). Han har laget en liknende værstasjon som det vi skal. Bakgrunnen for at han bygde sin værstasjon er at han bor på et område der det er dårlige representative værdata, så han valgte å ta saken i egne hender.

Han vurderte først å lage sin egen værstasjon fra bunnen, basert på enten på mikrokontrolleren Arduino eller Tessel, men fant ut at ingen av disse ville tåle påkjenningene av ekstremværet der han bor (Mount Washington). Derfor bestemte han seg for å kjøpe Davis 6250 Vantage Vue\cite{nettside:davis6250}, en ferdigbygd kombinert vind-, regn- og temperaturmåler. Denne værstasjonen sender data trådløst, og serveren som tar i mot dette er en Raspberry Pi model B\cite{nettside:rpi} med en Edimax EW-7811Un trådløs nettverksadapter\cite{nettside:edimax}.

Serveren programmerte han i Node.js, et JavaScript-bibliotek for nettverksservere. Han brukte også JavaScript for å få visualisert data på skjermer i huset. Han har også programmert et varslingssystem på serversiden ved bruk av Nodemailer som sender SMS-varslinger til telefonen hans når været når ekstreme verdier. Figur \ref{fig:drbunson} viser presentasjonen av værdata som er laget i JavaScript.
\begin{figure}[H]
  \centering
  \includegraphics[width=0.5\textwidth]{drbunson}
  \caption{Skjermdump av Dr. Bunsons JavaScript-presentasjon av værdata.}
  \label{fig:drbunson}
\end{figure}

\subsubsection{Arduino weather station\cite{nettside:instructables}}
En annen som har laget sin egen værstasjon, skrev en artikkel på hvordan han satt opp hele systemet, steg for steg. Han oppgir ikke navn på nettsiden, men han går under brukernavnet bram2202. Han bruker sensorer for trykk, temperatur, luftfuktighet og bevegelse. Han viser og skriver detaljert hva han bruker av utstyr og hvordan han koblet alt mot en Arduino mikrokontroller. Han har tatt med en sender og en mottaker mellom mikrokontrolleren og sensorene.

Værstasjonen ble bygd på en slik måte at han alltid kunne ha plass til utvidelser. Han viser ved et eksempel hvor han kobler opp og programmerer en sensor som måler nedbør. Alt han trengte var en måler for nedbør og en boks for å sikre denne måleren mot vann. Måleren kan bare kobles på en ledig port på mikrokontrolleren. Figur \ref{fig:arduino_vaerstasjon} viser værstasjonen etter den var ferdig oppkoblet.
\begin{figure}[H]
  \centering
  \includegraphics[width=0.5\textwidth]{arduinos_vaerstasjon}
  \caption{Den ferdige værstasjonen til bram2202 som viser temperatur, luftfuktighet og trykk}
  \label{fig:arduino_vaerstasjon}
\end{figure}

\subsection{Kan vi lære noe av disse prosjektene?}
Dr. Bunson mente at mikrokontrollerkort ikke tåler nok til å brukes som værstasjon der han bodde, men det var hovedsaklig på grunn av ekstrem vind. I Hessdalen er det meste sannsynligvis kulde som kan bli det største problemet, da temperaturen kan krype under 30 minusgrader\cite{nettside:yr150313}. Siden det er påkrevd i oppgaven at vi bruker Ethernut 2.1, må vi beskytte kortet godt. Med denne informasjonen i bakhodet, vil en prioritet for oss være å bygge vår værstasjon inn i en godt isolerende boks med termostat og varmelement. Grensesnittet Dr. Bunson har brukt for fremvisning av data kan også være en inspirasjon når vi skal utforme vår side for datapresentasjon.

Bram2202 forklarer ikke koden han har skrevet for prosjektet, men han har vedlagt det som en link som man kan laste ned med hans instillinger for værstasjonen hvis man ville lagd en helt liknende værstasjon. Det er kun kretsskjemaer og oppkoblingen av målerne han har forklart. Det vi kan bruke av dette prosjektet er at vi kan lage en liknende boks, der vi beskytter alle kretsene for vann.

\section{Værstasjonens historie}

\subsection{Fra observasjoner til datamodeller}
Vi har alltid vært avhengig av været, og behovet for å forutsi været har eksistert i lang tid. 
Været har alltid spilt en stor rolle for blant annet landbruk, fisking og skipsfart. Derfor har vi mennesker i årtusener hatt et ønske om å kunne forutsi hvordan været kommer til å bli.

I begynnelsen bygde forutsigelsene bare på observasjoner og tilfeldige målinger, men på 1700-tallet ble det utviklet instrumenter som la grunnlaget for at den moderne meteorologien kunne bli en reell vitenskap. Dermed kunne forskere forstå de faktorene som bestemmer værets utvikling, og lage nyttige værmeldinger som for eksempel stormvarsler.\cite{nettside:historienet1} Figur \ref{fig:himmel} viser en typisk storm der det kan være gunstig å måle diverse ting som nedbør, lufttrykk og temperatur.

\begin{figure}[H]
  \centering
  \includegraphics[width=0.75\textwidth]{himmel}
  \caption{Uvær der det kan være gunstig å måle forskjellige parametere som for eksempel nedbør og lufttrykk.}
  \label{fig:himmel}
\end{figure}

\subsection{Måling av lufttrykk}
Ingen visste at luft har tyngde før italieneren Evangelista Torricelli oppfant kvikksølvbarometeret i 1643. Han la merke til at en kvikksølvsøyle sto høyere da det for eksempel var sol og pent vær enn da det var dårlig vær. I 1714 fant Gabriel Fahrenheit ut at det gikk an å måle temperaturen i luft og vann med et kvikksølvtermometer. Det teoretiske grunnlaget ble også lagt på 1700-tallet, da sveitseren Daniel Bernoulli ga ut verket «Hydrodynamica». \cite{nettside:historienet2}

\subsection{Første ballongferd}
I 1783 gjennomførte Jacques Alexandre Charles den første flygningen med en luftballong som var fylt med hydrogen. Det viste seg at oppfinnelsen var svært velegnet til å samle inn meteorologiske data fra atmosfæren. Fordi det kunne være livsfarlig å sende opp luftballonger i atmosfæren med tanke på lyn og dårlig vær, sendte de første ballongskipperne opp små testballonger før de lettet selv.\cite{nettside:historienet3} Figur \ref{fig:ballongferd} viser Jaques Alexandre Charles sin ballongferd.

\begin{figure}[H]
  \centering
  \includegraphics[width=0.60\textwidth]{ballongferd}
  \caption{Figuren viser den andre ballongferden gjort av Jacques Alexandre Charles 1. desember 1793}
  \label{fig:ballongferd}
\end{figure}

\subsection{Utveksling av værdata}
Den irske admiralen Francis Beaufort foreslo i 1806 å dele vindstyrken i en skala på 12 trinn – som han senere kalte beaufortskalaen. Dette ga meteorologene en felles referanse. Da telegrafen ble oppfunnet senere på 1800-tallet, ble det for første gang praktisk mulig å innhente værdata fra større områder, og eksempelvis sende ut stormvarsler. I 1849 opprettet Smithsonian Institution et meteorologisk nettverk på tvers av USA, og lignende samarbeidsprosjekter mellom meteorologer ble etablert i Europa på resten av 1800-tallet.\cite{nettside:historienet4}


\subsection{Beregning av værdata}
I 1922 fremla den britiske meteorlogen Lewis Fry Richardson verket «Weather Prediction by Numerical Process» som var en rekke formler som gjorde det mulig å beregne været fremover i tid. Problemet var at det tok omtrent 24 timer å foreta alle de kompliserte utregningene. Dermed var morgendagen over før værmeldingen kunne være klar. Richardson hadde store tanker og ville samle 64 000 meteorologer på ett sted, slik at de kunne løse hver sin del av utregningene, men ideen ble aldri noe av. \cite{nettside:historienet5}

\subsection{Radiosonden et stort framskritt}

I 1927 sendte to forskere opp luftballonger med måleinstrumenter, utstyrt med en radiosender. Dette resulterte i at meteorologene fikk tilgang til bedre data. Målingene ble da sendt trådløst hjem, og verdens første radiosonde var en realitet. Ti år senere opprettet USAs meteorologiske institutt en fast værtjeneste basert på radiosonder i ballonger. Under andre verdenskrig ble radioteknikken utviklet videre, og midt i 1950-årene gjorde bedre materialer det mulig å sende opp større luftballonger enn noen gang. Radiosondene kunne nå komme opp i 50 kilometers høyde og samle inn svært nøyaktige data.\cite{nettside:historienet6} Figur \ref{fig:radiosonde} viser en typisk radiosonde.

\begin{figure}[H]
  \centering
  \includegraphics[width=0.65\textwidth]{radiosonde}
  \caption{Et eksempel på en radiosonde}
  \label{fig:radiosonde}
\end{figure}

\subsection{Første værsatellitt}
Da de første datamaskinene ble bygd på 1950-tallet, kunne meteorologene beregne værets utvikling et helt døgn frem. Men ettersom de tidligste datamaskinene bare hadde en brøkdel av kapasiteten til moderne datamaskiner, tok beregningene fremdeles nesten 24 timer. Fordi meteorologiske beregninger er så kompliserte, egner de seg for databehandling. Meteorologien var derfor med på å videreutvikle datateknologien.\cite{nettside:historienet7}

Fra 1960-årene fikk meteorologene tilgang på enda mer data, fordi de startet å sende opp flere værsatelitter. Den første, Vanguard 2, ble sendt opp av den amerikanske marinen i 1959. Den skulle måle skydekkets tykkelse, men den kom ut av kurs og ble ubrukelig.\cite{nettside:Vanguard2} Det gikk bedre med den neste værsatellitten, TIROS-1, sendt opp av NASA i 1960\cite{nettside:TIROS1}. Nå kunne all datainnsamling foregå automatisk. Figur \ref{fig:tiros_1} viser TIROS-1, den første vellykkende værsatellitten.

\begin{figure}[H]
  \centering
  \includegraphics[width=0.65\textwidth]{satellit}
  \caption{Den første vellykkede værsatellitten}
  \label{fig:tiros_1}
\end{figure}

\section{Måling av vær}

\subsection{Lufttrykk}
De meteorologiske målingene tar sikte på å finne den romlige fordelingen av det hydrostatiske trykk\cite{nettside:snl1}, som igjen bestemmer den storstilte luftsirkulasjonen. Lufttrykk angis i enheten pascal (forkortet Pa), men innen meteorologien brukes heller enheten hektopascal (hPa) som gir en tallverdi lik den tidligere enheten som brukes, millibar (mb). Lufttrykket blir angitt i atmosfærer eller mm kvikksølv (Hg). Én atmosfære (atm) er omtrent det lufttrykket som er ved havnivå. En atmosfære blir da 1013,25 hPa som tilsvarer 760 mm Hg (1000 hPa = 750 mm Hg). Apparater eller instrumenter for måling av lufttrykk kan være for eksempel et barometer. Figur \ref{fig:barometer} viser et enkelt barometer.

\begin{figure}[H]
  \centering
  \includegraphics[width=0.4\textwidth]{barometer}
  \caption{Et eksempel på et moderne barometer som måler lufttrykk}
  \label{fig:barometer}
\end{figure}

\subsection{Vind}

Vinden vil alltid variere. Derfor er internasjonal standard å måle middelverdien for vinden i 10 minutter, som også ligger til grunn ved bruk av Beauforts vindskala.

Vindens fart øker med høyden. Det kan være omtrent vindstille langs bakken, men ved noen få meters høyde kan det være kraftig vind. Vinden vil alltid bli større jo lengre opp man kommer, men etter et visst punkt vil hastigheten avta noe. Derfor er det er internasjonalt bestemt at vindmålinger for værvarslings- og klimaformål skal gjøres 10 meter over bakken. \cite{nettside:snl2}

\subsubsection{Vindfarten}
Vindfarten oppgis i m/s, knop eller km/h. Begrepet vindhastighet beskriver både fart og retning, men brukes ofte bare om fart.
Vindfarten bygger på Beauforts vindskala, som opprinnelig ble tatt i bruk på 1800-tallet av den britiske admiral Francis Beaufort. Skalaen ble i utgangspunket laget for seilfartøyer, og ble senere tilpasset instrumentmålinger av vindhastigheten. Vind angitt i skalaintervaller benevnes gjerne vindstyrke. I tabell \ref{table:beauforts} kan vi se en oppsummering av Beauforts vindskala. \cite{nettside:snl2}

\begin{table}
 \begin{tabular}{ | p{3cm} | p{3cm} | p{3cm} | p{3cm} | }
 \hline
     Beauforts skala & Navn på vindstyrken & Vindstyrke i knop & Vindstyrke i m/s \\ \hline
     0 & Stille & Mindre enn 1 & 0,0 - 0,2 \\ \hline
     1 & Flau vind & 1 - 3 & 0,3 - 1,5  \\ \hline
     2 & Svak vind & 4 - 6 & 1,6 - 3,3  \\ \hline
     3 & Lett bris & 7 - 10 & 3,4 - 5,4  \\ \hline
     4 & Laber bris & 11 - 16 & 5,5 - 7,9  \\ \hline
     5 & Frisk bris & 17 - 21 & 8,0 - 10,7  \\ \hline
     6 & Liten kuling & 22 - 27 & 10,8 - 13,8 \\ \hline
     7 & Stiv kuling & 28 - 33 & 13,9 - 17,1  \\ \hline
     8 & Sterk kuling & 34 - 40 & 17,2 - 20,7  \\ \hline
     9 & Liten storm & 41 - 47 & 20,8 - 24,4  \\ \hline
     10 & Full storm & 48 - 55 & 24,5 - 28,4  \\ \hline
     11 & Sterk storm & 56 - 63 & 28,5 - 32,6  \\ \hline
	 12 & Orkan & Over 63 & Over 32,6  \\ \hline
\end{tabular}
\caption{Beauforts vindskala}
\label{table:beauforts}
\end{table}

\subsection{Vindretning}

De to vanligste måtene å måle vindretningen på er med en vindpølse eller en vindfløy.

\subsubsection{Vindfløy}

Vindfløy er et instrument som viser vindretningen. Vanligvis er en vindfløy utført som en usymmetrisk figur som kan dreie om en vertikal akse, som da kan vise vindens retning. Vindfløyens areal er laget slik at den er større på den ene siden, den siden som er tyngst. Dermed vil denne siden svinge fra vinden. Svært ofte er det festet til et kors eller en sirkel med angivelse av himmelretninger på fløystangen.\cite{nettside:vindfloy} Figur \ref{fig:vindfloy} viser til en kjent vindfløy, som er plassert i Zeitz, Tyskland.

\begin{figure}[H]
  \centering
  \includegraphics[width=0.60\textwidth]{vindfloy}
  \caption{Vindfløy i smijern på industribygning i Zeitz, Tyskland }
  \label{fig:vindfloy}
\end{figure}

\subsubsection{Vindpølse}

En vindpølse er gjerne en stor konisk-formet sekk som er laget av tekstil. Vindpølsa er åpen i begge ender og den viser vindretning og vindhastighet. Vindpølser er mest brukt på flyplasser, helikopterplasser og på bruer.\cite{nettside:vindpolse} Figur \ref{fig:vindfloy} viser en vindpølse på en liten, lokal flyplass.\\

\begin{figure}[H]
  \centering
  \includegraphics[width=0.60\textwidth]{vindpolse}
  \caption{En typisk vindpølse plassert på en liten flyplass}
  \label{fig:vindfloy}
\end{figure}

\textbf{Vindpølsers generelle krav:}\cite{nettside:vindpolse}
\begin{itemize}
\item Vindpølsa skal gi presis indikasjon på vindretning og grov indikasjon på vindstyrke.
\item Vindpølsa skal være traktformet og åpen i begge ender.
\item Vindpølsa skal være oransje eller oransje og hvit. Dersom oransje og hvit velges, skal fargene være i form av 5 sirkulære bånd der første og siste bånd er oransje.
\end{itemize}
\textbf{I tillegg for flyplasser:}
\begin{itemize}
\item Lengden på vindpølsa skal være minst 3,6 meter.
\item Vindpølsas største åpningen skal ikke være mindre enn 0,9 m, og den minste åpningen ikke mindre enn 0,3 m.
\end{itemize}
\textbf{I tillegg for helikopterplasser:}
\begin{itemize}
\item Lengden på vindpølsa skal være minst 2,4 meter,
\item Vindpølsas største åpningen skal ikke være mindre enn 0,6 m, og den minste åpningen ikke mindre enn 0,3 m,
\end{itemize}






























