\subsection{Værstasjon i Hessdalen}
\subsubsection{Bakgrunn}
\begin{figure}[h!]
  \centering
  \includegraphics[width=0.5\textwidth]{lysfen}
  \caption{Bilde tatt av Arne P. Thomassen 25. Oktober 1982, mellom klokken 19.00 og 20.30, fra Finnsåhøgda syd, mot øst.}
\end{figure}
I 1981 ble det observert flere lys flytende over himmelen i en bygd ved navn Hessdalen i Holtålen kommune, Sør-Trøndelag. De ble beskrevet som "brennende ildkuler", og fenomenet ble vidt omtalt i medier. I 1981-1984 var det på det meste mellom 15-20 observasjoner i uka, og folk begynte å omtale det som "UFO-observasjoner". Dette førte til at Hessdalen ble et attraktivt turistmål for de som ønsket å se fenomenet. Jevnligheten på observasjonene har sunket betraktelig etter dette, og er nå nede i 10-20 i året. \cite{nettside:hessdalen}
\subsubsection{Project Hessdalen}
"Project Hessdalen" er et prosjekt startet i 1983 av organisasjonen "UFO-Norge" for å finne ut mer om de stadig uforklarte lysfenomenene som blir observert i Hessdalen. I dag blir prosjektet ledet av Erling P. Strand og Høgskolen i Østfold. Det ble satt opp en automatisk målestasjon i 1998 for å filme fenomenene når det oppstår.
\subsubsection{Værstasjon}
I forbindelse med å utvide den automatiske målestasjonen, går prosjektet vårt ut på å bygge og sette opp to værstasjoner som skal innhente værdata regelmessig og lagre dette i en database på Høgskolen i Østfold.  De to enhetene skal monteres på to bestemte punkter 171 meter fra hverandre. Deretter skal innhentet værdata presenteres visuelt på www.hessdalen.org.\newline Etter ønske fra oppdragsgiver skal følgene data logges på timesbasis:
\begin{compactitem}
\item [{\bf Temperatur}] Maksimum, minimum, gjennomsnitt, nåværende, tidspunkt for maksimum og tidspunkt for minimum.
\item [{\bf Luftfuktighet}] Maksimum, minimum, gjennomsnitt, nåværende, tidspunkt for maksimum og tidspunkt for minimum.
\item [{\bf Lufttrykk}] Maksimum, minimum, gjennomsnitt, nåværende, tidspunkt for maksimum og tidspunkt for minimum.
\item [{\bf Vind}] Maksimum, minimum, gjennomsnitt, nåværende, vindretning for maksimum, tidspunkt for maksimum og tidspunkt for minimum.
\end{compactitem}
\subsection{Krav}
Oppdragsgivers krav er at vi skal ha minst én operativ værstasjon i Hessdalen der minst én til to typer for værdata blir logget. Værstasjonene må bygges med uP-kortet (Ethernut 2.1) som mikrokontroller. Data skal være visuelt presentert på www.hessdalen.org, og gi mulighet for brukerene å velge et gitt tidsintervall de vil se data fra. Data må kunne presenteres på både kurveform og tabellform. 
