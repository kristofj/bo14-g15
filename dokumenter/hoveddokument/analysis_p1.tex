\subsection{Grundig beskrivelse av oppgaven basert på skissen gitt av oppdragsgiver (trenger ny tittel)}
Oppdragsgiver ønsker to værstasjoner for jevnlig innhenting og rapportering av været i Hessdalen. Disse værstasjonene skal bygges og settes opp i forbindelse med "Project Hessdalen".  

\subsubsection{Bakgrunn}
I 1981 ble det observert flere lys flytende over himmelen i en bygd ved navn Hessdalen i Holtålen kommune, Sør-Trøndelag. De ble beskrevet som "brennende ildkuler", og fenomenet ble vidt omtalt i medier. I 1981-1984 var det mellom 15-20 observasjoner i uka, og folk begynte å omtale det som "UFO-observasjoner". Dette førte til at Hessdalen ble et attraktivt turistmål for de som ønsket å se fenomenet. Jevnligheten på observasjonene har sunket betraktelig etter dette, og er nå nede i 10-20 i året. 
\subsubsection{Project Hessdalen}
Project Hessdalen" er et prosjekt startet i 1983 av organisasjonen "UFO-Norge" for å finne ut mer om de stadig uforklarte lysfenomenene som blir observert i Hessdalen. I dag blir prosjektet ledet av Erling P. Strand og Høgskolen i Østfold.
\subsubsection{Værstasjon}
Prosjektet vårt går ut på å bygge og sette opp to værstasjoner i Hessdalen som skal innhente værdata regelmessig og lagre dette i en database på Høgskolen i Østfold. Deretter skal innhentet værdata presenteres visuelt på www.hessdalen.org. Etter krav fra oppdragsgiver skal følgene data logges på timesbasis:
\begin{compactitem}
\item [{\bf Temperatur}] Maksimum, minimum, gjennomsnitt, nåværende, tidspunkt for maksimum og tidspunkt for minimum.
\item [{\bf Luftfuktighet}] Maksimum, minimum, gjennomsnitt, nåværende, tidspunkt for maksimum og tidspunkt for minimum.
\item [{\bf Lufttrykk}] Maksimum, minimum, gjennomsnitt, nåværende, tidspunkt for maksimum og tidspunkt for minimum.
\item [{\bf Vind}] Maksimum, minimum, gjennomsnitt, nåværende, vindretning for maksimum, tidspunkt for maksimum og tidspunkt for minimum.
\end{compactitem}