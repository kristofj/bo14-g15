\def\category{Bacheloroppgave}
\def\ects{20}
\def\area{Informasjonsteknologi}
\def\free{X}
\def\freeafter{30/12 2029}
\def\freecustomer{X}

\def\customer{Erling Petter Strand}
\def\tutor{Einar Von Krogh}
\def\department{Avdeling for Informasjonsteknologi (alle programmer)}
\def\projectnr{BO14-G15}
\def\contact{Erling Petter Strand}
\def\abstract{
Vår bacheloroppgave er å lage to værstasjoner som skal måle temperatur, lufttrykk, vindhastighet og vindretning. Disse dataene skal lagres på en server på skolen. For deretter å bli visualisert på hjemmesiden til Project Hessdalen. Rapporten inneholder en introduksjon til gruppa og oppgaven vår. Deretter vil vi beskrive hvordan vi har planlagt arbeidet, hvordan utformingen av produktet gikk, hvordan vi har testet produktet vårt og til slutt en refleksjon på hvordan vi syns det har gått. Hva som kunne vært bedre og hva vi er fornøyde med.  }
\def\keyone{Ukjent\dots}
\def\keytwo{Ukjent \dots}
\def\keythree{Ukjent \dots}
\author{Mikael Johansen Grimstad, Kristoffer Jensen, Kristian Norum Karlsen og Morten Lindstad}
\title{Værstasjon i Hessdalen}
\date{\today
}