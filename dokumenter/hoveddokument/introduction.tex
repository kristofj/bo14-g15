%\cleardoublepage
\chapter{Introduksjon}
\label{chap:intro}

\section{Prosjektgruppen}

Mikael Johansen Grimstad(født 23.08.1991) gikk tidligere studiespesialisering med realfag ved Frederik II VGS(2007-2010). I året som fulgte utførte han verneplikt som gardist ved Hans Majestet Kongens Garde. Videre valgte å studere Informatikk - Design og utvikling av IT-systemer ved Høgskolen i Østfold fordi han alltid har hatt en stor interesse for elektronikk, data og programmering.\\

Kristian Norum Karlsen(født 27.09.1991) gikk tidligere studiespesialisering med realfag ved Frederik II VGS(2007-2010). I året som fulgte utførte han verneplikt som sambandsmann i Sambandsbataljonen. Videre valgte han bachelorstudiet dataingeniør ved Høgskolen i Østfold som han fikk et innblikk i under et elevbesøk i 2. klasse på videregående. Dataingeniørstudiet falt som et naturlig valg da dette tar for seg mange av hans interesser.\\

Kristoffer Jensen(født 14.11.1991) gikk tidligere studiespesialiserende elektrofag ved Glemmen VGS(2007-2010). I året som fulgte utførte han verneplikt som luftvernartillerist ved Ørland hovedflystasjon. Videre valgte han bachelorstudiet dataingeniør ved Høgskolen i Østfold da dette virket som et naturlig valg med tanke på hans interesser.\\

Morten Lindstad(født 10.05.1992) gikk tidligere på elektrofag med studiekompetanse(2008-2011) ved Glemmen VGS før han startet på dataningeniør på Høgskolen i Østfold. Dette viste seg å ha flere fordeler med tanke på at flere av fagene går igjen med hva han hadde på videregående. Han har alltid hatt en stor interesse for elektronikk og data, alt fra elektriske gitarer til dataspill. Dette er også en fordel fordi han og vil mest sannsynlig ha en enklere forståelse for bacheloroppgaven.\\

Det var ikke noe tilfeldighet at det var akkurat vi fire som havnet på samme gruppe. Vi har jobbet som en gruppe i flere annledninger tidligere. Vi har samarbeidet i fag som Datakommunikasjon, Bildebehandling og mønstergjenkjenning og Integrerte IT-systemer. 

\section{Oppdragsgiver}

Oppdragsgiver er Erling Petter Strand, høgskolelektor ved avdeling for informasjonsteknologi på Høgskolen i Østfold. Han er sivilingeniør fra Norges teknisk-vitenskapelige universitet i Trondheim innenfor elektro med studieretning fysikalsk elektronikk og teleteknikk.
Han har jobbet for Standard Telefon og Kabelfabrikk, EDAS målesystemer og EDB- og Automatiseringsavdelingen på Østfold Ingeniørhøgskole  i Sarpsborg som ble sammenslått med informatikk i Halden som nå er avdeling for informasjonsteknologi på Høgskolen i Østfold.
Han er en av grunnleggerene av Project Hessdalen og er nå leder av dette prosjektet. 

\section{Oppdraget}
\label{sec:oppgaven}

Oppdragsgiver ønsker to værstasjoner som skal kunne måle temperatur, lufttrykk, luftfuktighet, vindhastighet og vindretning. Disse to stasjonene skal kobles opp på to forskjellige steder og værdataene skal sendes og lagres på serveren til Høgskolen i Østfold for deretter å bli presentert på hjemmesiden til Project Hessdalen.

\section{Formål, leveranser og metode}
\label{sec:maal-metode-resultater}

\subsection{Formål}
\label{sec:maal}

\begin{compactenum}[{\bf Hovedmål} \bf :]
\item Lage en komplett værstasjon med datapresentasjon på hjemmesiden til Project Hessdalen.
\begin{compactenum}[{\bf  Delmål} \bf 1:]
\item Lage en værstasjon som skal måle temperatur, lufttrykk, luftfuktighet, vindretning og vindhastighet.
\item Lagre værdata på server til Høgskolen i Østfold.
\item Presentere værdata på hjemmesiden til Project Hessdalen.
\item Tilrettelegge for å kunne sammenligne værdata fra ulike tidsperioder.
\item Lage et kabinett til \glslink{mikrokontroller}{mikrokontrolleren}, koblingsbrettet og strømforsyningene.
\item Tilpassning av hardware. Med tanke på lavpassfilter, spenningsdelere, festeanordninger for sensorer og kabinettet.
\end{compactenum}
\end{compactenum}

\subsection{Leveranser}
\label{sec:resultater}
Resultatet av prosjektet vil være to værstasjoner med datapresentasjon og en rapport. Dataene som værstasjonene produserer vil være tilgjengelig på hjemmesiden til Project Hessdalen. Rapporten vil være en dokumentasjon av prosjektet. 

\subsection{Metode}
\label{sec:metode}
Slik vi jobber går dette under den induktive metode, det vil si at vi lærer av å gjøre. Det vi vet er at \glslink{mikrokontroller}{mikrokontrolleren}, Ethernut 2.1, skal brukes til å lese sensorene. Det vi må finne ut er hvilke sensorer som kan brukes på \glslink{mikrokontroller}{mikrokontrolleren}. Dette gjøres ved å lese databladene til de sensorene vi tror er aktuelle. Lage en liste over de ulike kandidatene og sende denne til arbeidsgiver som velger ut de som passer best. Når værstasjonen er ferdig skal den testes på skolen før vi monterer den i Hessdalen. God arbeidsfordeling er viktig for oss siden vi er en gruppe på fire. Effektiv jobbing blir viktig for å klare å utføre alle arbeidsoppgavene.

Denne prosjektrapporten skal bli skrevet ved hjelp av \LaTeX. Det er typesettingssystem for dokumentproduksjon. All kildekode og tekst kommer vi så til å legge på GitHub, som er en web-basert tjeneste for versjonskontroll og lagring av programvareprosjekter. Da vil alle i gruppen ha tilgang til nyeste versjon, og alle vil ha mulighet til å se og endre alt i prosjektet.

Da vi skal dele inn gruppen slik at alle får forskjellige ansvarsområder, har vi valgt å bruke en flat struktur. Sekretærrollen er den eneste rollen vi skal benytte, den skal gå på omgang blandt alle gruppemedlemmene. Sekretæren har ansvar for å skrive ned møtereferat og ukereferat.
\section{Rapportstruktur}

I kapittel~\ref{chap:analysis} starter vi med bakgrunnsinformasjon på fenomenene til de brennende ildkulene og Project Hessdalen. Deretter mer informasjon om værstasjonen med spesifikke krav. Alt rundt værstasjonen må planlegges og dette tar vi for oss  i kapittel~\ref{chap:planning}. Her forklarer vi også hvordan hele systemet fungerer. Fra værdataene kommer inn til \glslink{mikrokontroller}{mikrokontrolleren} og til de er tilgjengelig på nettsiden til Project Hessdalen. Hvordan selve produksjonen foregikk forklarer vi i kapittel~\ref{chap:production}. Deretter skal systemet testes og godkjennes før det skal monteres i målingstasjonen til Project Hessdalen. Hvordan dette ble gjort blir forklart i kapittel~\ref{chap:evaluation}. Tilslutt har vi en diskusjonsfase, kapittel~\ref{chap:discussion}, hvor vi tar opp hva vi har lært, problemer vi har støtt på, hva vi er fornøyd med, ble arbeidsgiver fornøyd og hva ville vi har gjort annerledes dersom vi skulle ha gjort prosjektet på nytt. I kapittel \ref{chap:conclusion} tar vi opp de viktigste punktene fra diskusjonskapittelet og hva som burde gjøres dersom dette produktet skal videreutvikles.