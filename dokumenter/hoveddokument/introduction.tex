%\cleardoublepage
\chapter{Introduksjon}
\label{chap:intro}
\meta{
Det som er markert med grått, er forklaringer på hva de enkelte delene av rapportene skal inneholde, og som er det minimum leseren bør skumme gjennom før malen taes i bruk. Det som ikke er markert med grått, er eksempeltekst som kunne tenkes brukt i et fiktivt prosjekt der formålet er å utarbeide en mal for hoveddokumentet i en bacheloroppgave ved HiØ/IT.
Introduksjonen skal gi leseren et bilde av rammene rundt prosjektet, prosjektets formål, metoder og leveranser. Den bør også inneholde en oversikt over resten av dokumentet[1]. Husk at kapitler, sections etc. bør ha et par setninger med ``innledning'' før man starter på neste undernivå.
}

\section{Prosjektgruppen}

\meta{
\textit{Det er vanlig å starte med å presentere prosjektgruppen, litt om hver enkelt av deltagerne, deres kompetanse og interesser, og litt om hvordan dere har kommet sammen, f.eks. om dere har jobbet sammen i andre fag.}
}
Mikael Johansen Grimstad(født 23.08.1991) gikk tidligere studiespesialisering med realfag ved Frederik II VGS(2007-2010). I året som fulgte utførte han verneplikt som gardist ved Hans Majestet Kongens Garde. Videre valgte å studere Informatikk - Design og utvikling av IT-systemer ved Høgskolen i Østfold fordi han alltid har hatt en stor interesse for elektronikk, data og programmering.\\

Kristian Norum Karlsen(født 27.09.1991) gikk tidligere studiespesialisering med realfag ved Frederik II VGS(2007-2010). I året som fulgte utførte han verneplikt som sambandsmann i Sambandsbataljonen. Videre valgte han bachelorstudiet dataingeniør ved Høgskolen i Østfold som han fikk et innblikk i under et elevbesøk i 2. klasse på videregående. Dataingeniørstudiet falt som et naturlig valg da dette tar for seg mange av hans interesser.\\

Kristoffer Jensen(født 14.11.1991) gikk tidligere studiespesialiserende elektrofag ved Glemmen VGS(2007-2010). I året som fulgte utførte han verneplikt som luftvernartillerist ved Ørland hovedflystasjon. Videre valgte han bachelorstudiet dataingeniør ved Høgskolen i Østfold da dette virket som et naturlig valg med tanke på hans interesser.\\

Morten Lindstad(født 10.05.1992) gikk tidligere på elektrofag med studiekompetanse(2008-2011) ved Glemmen VGS før han startet på dataningeniørlinja på Høgskolen i Østfold. Dette viste seg å ha flere fordeler med tanke på at flere av faga går igjen med hva han hadde på videregående. Han har alltid hatt en stor interesse for elektronikk og data, alt fra elektriske gitarer til dataspill. Dette er også en fordel fordi han og vil mest sannsynlig ha en enklere forståelse for bacheloroppgaven.\\

Det var ikke noe tilfeldighet at det var akkurat vi fire som havnet på samme gruppe. Vi har jobbet som en gruppe i flere annledninger tidligere. Vi har samarbeidet i fag som Datakommunikasjon, Bildebehandling og mønstergjenkjenning og Integrerte IT-systemer. 

\section{Oppdragsgiver}

\meta{
\textit{Beskriv oppdragsgiver, både firma og kontaktpersoner.}
}
Vår oppdragsgiver er Erling Petter Strand, høgskolelektor ved avdeling for informasjonsteknologi på Høgskolen i Østfold. Han er sivilingeniør fra Norges teknisk-vitenskapelige universitet i Throndheim innenfor elektro med studieretning fysikalsk elektronikk og teleteknikk.
Han har jobbet for Standard Telefon og Kabelfabrikk, EDAS målesystemer og EDB- og Automatiseringsavdelingen på Østfold Ingeniørhøgskole  i Sarpsborg som ble sammenslått med informatikk i Halden som nå er avdeling for informasjonsteknologi på Høgskolen i Østfold.
Han er en av grunnleggerene av Project Hessdalen og er nå leder av dette prosjektet. 

\section{Oppdraget}
\label{sec:oppgaven}

\meta{
\textit{Forklar hva slags problem oppdragsgiver ønsker å løse, og hvordan det er tenkt gjort. Dette skal ikke være en inngående analyse, men en utvidet og oppdatert versjon av det som opprinnelig fantes i prosjektbeskrivelsen, f.eks. slik:}
}
Oppdragsgiver ønsker to værstasjoner som skal kunne måle temperatur, lufttrykk, luftfuktighet, vindhastighet og vindretning. Disse to stasjonene skal kobles opp på to forskjellige steder og værdataene skal sendes og lagres på serveren til Høgskolen i Østfold for deretter å bli presentert på hjemmesiden til Project Hessdalen.

\section{Hvorfor, hva og hvordan: Formål, leveranser og metode}
\label{sec:maal-metode-resultater}
\meta{
I dette kapittelet tar dere for dere tre sentrale aspekter: Formål, leveranser og metode. {\em Formålet} (ofte bare kalt {\em målet}, skal beskrive virkningen av prosjektet på et overordnet plan (f.eks. øke omsetningen i et firma). 
{\em Leveransene} er konkrete resultater (tangibles) som blir produsert underveis (f.eks. programvare med tilhørende brukerdokumentasjon), mao.\ {\em hva} som skal produseres. 
{\em Metoden} er {\em hvordan} formål og leveranser skal oppnås (f.eks. analysere dagens situasjon og designe og utvikle en ny nettbutikk). 
Jo mer teoretisk og ``akademisk'' prosjektet er, jo større vekt må man legge på metoden. Tradisjonelt er det metodiske aspektet relativt nedtonet i en bacheloroppgave i forhold til et master- eller PhD-prosjekt.
Erfaringsmessig oppfatter studentene dette som en litt fremmed måte å betrakte et prosjekt på, men den er utbredt i både akademia og næringslivet, og gjør det lettere å holde tunga rett i munnen underveis. 
Formålet uttrykkes gjerne som ett hovedmål, og et par-tre delmål som utdyper hovedmålet. Beskrivelsen av et mål starter nesten alltid med et verb. I dette prosjektet kan formål, resultater og metode beskrives slik:
}

\subsection{Formål}
\label{sec:maal}

\begin{compactenum}[{\bf Hovedmål}]
\item Lage en komplett værstasjon med datapresentasjon på hjemmesiden til Project Hessdalen.
\begin{compactenum}[{\bf  Delmål} \bf 1]
\item Lage en værstasjon som skal måle temperatur, lufttrykk, luftfuktighet, vindretning og vindhastighet.
\item Lagre værdata på server til Høgskolen i Østfold.
\item Presentere værdata på hjemmesiden til Project Hessdalen.
\item Tilrettelegge for å kunne sammenligne værdata fra ulike tidsperioder.
\end{compactenum}
\end{compactenum}

\subsection{Leveranser}
\label{sec:resultater}
Hovedresultatet av prosjektet vil være to værstasjoner og en rapport. Dataene som værstasjonene produserer vil være tilgjengelig på hjemmesiden til Project Hessdalen. Rapporten vil være en dokumentasjon av prosjektet. Hvordan  

\subsection{Metode}
\label{sec:metode}
For å oppnå dette vil det bli utført en ad-hoc undersøkelse av krav og retningslinjer til bacheloroppgaver ved andre høgskoler og universiteter, både nasjonalt og internasjonalt. I tillegg vil forfatterens erfaring som sensor og veileder av bacheloroppgaver bli brukt som utgangspunkt for å lage en anbefalt struktur med tilhørende innhold.
Videre vil det bli gjort en studie av omtaler og beskrivelser av verktøy for produksjon, vedlikehold og publisering av store og komplekse dokumenter. Det vil bli lagt spesiell vekt på muligheter og erfaringer med å bruker maler. I tillegg vil sentral og kritisk funksjonalitet bli utprøvet ved de mest lovende verktøyene. På denne bakgrunnen (samt forfatterens egne erfaringer) vil det bli valgt ut to verktøy. Verktøyene vil til slutt bli brukt til å lage to maler, som tilbyr mer eller mindre samme funksjonalitet, og mer eller mindre likt utseende resultat. Malene skal utformes slik at de inneholder all nødvendig informasjon, både når det gjelder hvordan de skal brukes, hvordan rapporten skal struktureres, og hva slags innhold de enkelte delene skal ha.

Slik vi jobber går dette innunder den induktive metode, det vil si at vi lærer av å gjøre. Det vi vet er at mikroprosessoren Ethernut 2.0 skal brukes til å lese sensorene. Det vi må finne ut er hvilke sensorer som kan brukes på Ethernut 2.0. Dette gjøres ved å lese databladene til de sensorene  vi finner og tror er aktuelle. Lage en liste over de ulike kandidatene og sende denne til arbeidsgiver som velger ut de som passer best. Når værstasjonen er ferdig skal den testes på skolen før vi monterer den i Hessdalen. God arbeidsfordeling er viktig for oss siden vi er en gruppe på fire. Effektiv jobbing blir viktig for å klare å utføre alle arbeidsoppgavene.....
\section{Rapportstruktur}

\meta{
Det er vanlig å avslutte innledningen med en oversikt over resten av rapporten, f.eks. slik som dette:
}

I Kapittel~\ref{chap:analysis} starter vi med å se på generelle krav til akademiske og tekniske dokumenter, og ikke minst hva som skiller disse fra ``vanlige'' dokumenter. Vi ser deretter nærmere på krav og retningslinjer til bachelorrapporter ved nasjonale og internasjonale læresteder. Vi går også gjennom HiØ/IT sine egen beskrivelse av hovedprosjektet.  Vi gir eksempler på maler fra andre læresteder. Deretter ser vi på de tekniske sidene ved å produsere store og komplekse dokumenter, med spesiell vekt på aktuelle programvareverktøy. Vi presenterer en overordnet design av vår rapportmal i Kapittel~\ref{chap:design}, og beskriver den konkrete implementasjon i OpenOffice i Kapittel~\ref{chap:implementation}. Løsningen blir ad-hoc evaluert i Kapittel~\ref{chap:evaluation}, og i Kapittel~\ref{chap:discussion} diskuterer vi resultatet av prosjektet. Rapporten avsluttes med en kort konklusjon i Kapittel~\ref{chap:conclusion}. En mer detaljert gjennomgang av hvordan malen kan brukes finnes i Vedlegg~ \ref{chap:how-to}.


