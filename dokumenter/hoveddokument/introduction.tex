%\cleardoublepage
\chapter{Introduksjon}
\label{chap:intro}

\section{Prosjektgruppen}

Mikael Johansen Grimstad(født 23.08.1991) gikk tidligere studiespesialisering med realfag ved Frederik II VGS(2007-2010). I året som fulgte utførte han verneplikt som gardist ved Hans Majestet Kongens Garde. Videre valgte å studere Informatikk - Design og utvikling av IT-systemer ved Høgskolen i Østfold fordi han alltid har hatt en stor interesse for elektronikk, data og programmering.\\

Kristian Norum Karlsen(født 27.09.1991) gikk tidligere studiespesialisering med realfag ved Frederik II VGS(2007-2010). I året som fulgte utførte han verneplikt som sambandsmann i Sambandsbataljonen. Videre valgte han bachelorstudiet dataingeniør ved Høgskolen i Østfold som han fikk et innblikk i under et elevbesøk i 2. klasse på videregående. Dataingeniørstudiet falt som et naturlig valg da dette tar for seg mange av hans interesser.\\

Kristoffer Jensen(født 14.11.1991) gikk tidligere studiespesialiserende elektrofag ved Glemmen VGS(2007-2010). I året som fulgte utførte han verneplikt som luftvernartillerist ved Ørland hovedflystasjon. Videre valgte han bachelorstudiet dataingeniør ved Høgskolen i Østfold da dette virket som et naturlig valg med tanke på hans interesser.\\

Morten Lindstad(født 10.05.1992) gikk tidligere på elektrofag med studiekompetanse(2008-2011) ved Glemmen VGS før han startet på dataningeniørlinja på Høgskolen i Østfold. Dette viste seg å ha flere fordeler med tanke på at flere av faga går igjen med hva han hadde på videregående. Han har alltid hatt en stor interesse for elektronikk og data, alt fra elektriske gitarer til dataspill. Dette er også en fordel fordi han og vil mest sannsynlig ha en enklere forståelse for bacheloroppgaven.\\

Det var ikke noe tilfeldighet at det var akkurat vi fire som havnet på samme gruppe. Vi har jobbet som en gruppe i flere annledninger tidligere. Vi har samarbeidet i fag som Datakommunikasjon, Bildebehandling og mønstergjenkjenning og Integrerte IT-systemer. 

\section{Oppdragsgiver}

Vår oppdragsgiver er Erling Petter Strand, høgskolelektor ved avdeling for informasjonsteknologi på Høgskolen i Østfold. Han er sivilingeniør fra Norges teknisk-vitenskapelige universitet i Throndheim innenfor elektro med studieretning fysikalsk elektronikk og teleteknikk.
Han har jobbet for Standard Telefon og Kabelfabrikk, EDAS målesystemer og EDB- og Automatiseringsavdelingen på Østfold Ingeniørhøgskole  i Sarpsborg som ble sammenslått med informatikk i Halden som nå er avdeling for informasjonsteknologi på Høgskolen i Østfold.
Han er en av grunnleggerene av Project Hessdalen og er nå leder av dette prosjektet. 

\section{Oppdraget}
\label{sec:oppgaven}

Oppdragsgiver ønsker to værstasjoner som skal kunne måle temperatur, lufttrykk, luftfuktighet, vindhastighet og vindretning. Disse to stasjonene skal kobles opp på to forskjellige steder og værdataene skal sendes og lagres på serveren til Høgskolen i Østfold for deretter å bli presentert på hjemmesiden til Project Hessdalen.

\section{Hvorfor, hva og hvordan: Formål, leveranser og metode}
\label{sec:maal-metode-resultater}

\subsection{Formål}
\label{sec:maal}

\begin{compactenum}[{\bf Hovedmål}]
\item Lage en komplett værstasjon med datapresentasjon på hjemmesiden til Project Hessdalen.
\begin{compactenum}[{\bf  Delmål} \bf 1]
\item Lage en værstasjon som skal måle temperatur, lufttrykk, luftfuktighet, vindretning og vindhastighet.
\item Lagre værdata på server til Høgskolen i Østfold.
\item Presentere værdata på hjemmesiden til Project Hessdalen.
\item Tilrettelegge for å kunne sammenligne værdata fra ulike tidsperioder.
\end{compactenum}
\end{compactenum}

\subsection{Leveranser}
\label{sec:resultater}
Hovedresultatet av prosjektet vil være to værstasjoner og en rapport. Dataene som værstasjonene produserer vil være tilgjengelig på hjemmesiden til Project Hessdalen. Rapporten vil være en dokumentasjon av prosjektet. Hvordan  

\subsection{Metode}
\label{sec:metode}
Slik vi jobber går dette innunder den induktive metode, det vil si at vi lærer av å gjøre. Det vi vet er at mikroprosessoren Ethernut 2.0 skal brukes til å lese sensorene. Det vi må finne ut er hvilke sensorer som kan brukes på Ethernut 2.0. Dette gjøres ved å lese databladene til de sensorene  vi finner og tror er aktuelle. Lage en liste over de ulike kandidatene og sende denne til arbeidsgiver som velger ut de som passer best. Når værstasjonen er ferdig skal den testes på skolen før vi monterer den i Hessdalen. God arbeidsfordeling er viktig for oss siden vi er en gruppe på fire. Effektiv jobbing blir viktig for å klare å utføre alle arbeidsoppgavene.....
\section{Rapportstruktur}

I Kapittel~\ref{chap:analysis} starter vi med å se på generelle krav til akademiske og tekniske dokumenter, og ikke minst hva som skiller disse fra ``vanlige'' dokumenter. Vi ser deretter nærmere på krav og retningslinjer til bachelorrapporter ved nasjonale og internasjonale læresteder. Vi går også gjennom HiØ/IT sine egen beskrivelse av hovedprosjektet.  Vi gir eksempler på maler fra andre læresteder. Deretter ser vi på de tekniske sidene ved å produsere store og komplekse dokumenter, med spesiell vekt på aktuelle programvareverktøy. Vi presenterer en overordnet design av vår rapportmal i Kapittel~\ref{chap:design}, og beskriver den konkrete implementasjon i OpenOffice i Kapittel~\ref{chap:implementation}. Løsningen blir ad-hoc evaluert i Kapittel~\ref{chap:evaluation}, og i Kapittel~\ref{chap:discussion} diskuterer vi resultatet av prosjektet. Rapporten avsluttes med en kort konklusjon i Kapittel~\ref{chap:conclusion}. En mer detaljert gjennomgang av hvordan malen kan brukes finnes i Vedlegg~ \ref{chap:how-to}.


