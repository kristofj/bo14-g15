%\cleardoublepage
\chapter{Diskusjon}
\label{chap:discussion} 

Her kommer vil til å ta opp hva vi har lært av prosjektet. Se om vi klarte å nå målene fra avsnitt \ref{sec:maal}, leverte vi det forventede resultatet i avsnitt \ref{sec:resultater} og fungerte metoden i avsnitt \ref{sec:metode}. Dertetter diskutere hva vi er fornøyde med, hva som kunne vært bedre og forklare hvorfor det ble bra eller dårlig. Ta opp eventuelle problemer vi har støtt på og tilslutt si noen ord om hva vi ville ha gjort anderledes om vi skulle ha gjort dette på nytt.


\section{Erfaringer vi har gjort oss}

Gruppen har tilegnet seg mye kunnskap. Vi har fått jobbet med utvikling av maskinvare, programvare og prøvd oss i feltarbeid. Prosessen har ført til at vi nå sitter med ny kunnskap om hvordan prosjekter bør planlegges og utføres.

\subsection{Måloppnåelse}

Dette er målene fra kapittel \ref{chap:intro}.

\begin{compactenum}[{\bf Hovedmål}]
\item Lage en komplett værstasjon med datapresentasjon på hjemmesiden til Project Hessdalen.
\begin{compactenum}[{\bf  Delmål} \bf 1]
\item Lage en værstasjon som skal måle temperatur, lufttrykk, luftfuktighet, vindretning og vindhastighet.
\item Lagre værdata på server til Høgskolen i Østfold.
\item Presentere værdata på hjemmesiden til Project Hessdalen.
\item Tilrettelegge for å kunne sammenligne værdata fra ulike tidsperioder.
\item Lage et kabinett til mikrokontrolleren, koblingsbrettet og strømforsyningene.
\item Tilpassning av hardware. Med tanke på lavpassfilter, spenningsdelere.
\end{compactenum}
\end{compactenum}

Vi nådde alle målene vi satt oss i planleggningsfasen. 

\subsection{Opp til forventningene?}
\label{subsec:opptilforventningene}
Resultatet vi leverte levde opp til både våre egne og oppdragsgivers satte mål. Under en samtale vi hadde med han spurte vi om han er fornyød med prosjektet. Han sa seg meget godt fornyød med arbeidet vi hadde gjort, spesielt med tanke på de spesifikasjoner som ble gitt oss i forkant av prosjektet. 

\subsection{Arbeidsmetoden}

Den induktive metoden fungerte. Siden vi har begrenset erfaring innefor værstasjoner og hvordan disse skal lages, ble det til at vi måtte lære oss mye underveis. Hvordan sensorene kommuniserer med mikrokontrolleren var en av tingene vi måtte tilegne oss kunnskaper om på grunn av manglene erfaring på området. Derav har den indutive metoden fungert bra. 

\section{Fornøyd, misfornøyd?}

\subsection{Hva vi er fornøyde med}
Når vi ser på systemet i sin helhet er vi meget fornøyd med det vi har fått til. Værstasjonene er montert og har vært i drift i snart to uker. I denne perioden har det ikke vært noen problemer med hverken avlesninger, sendinger eller visualisering av data.      

Vi er meget fornøyd hvor variert oppgaven har vært. Den har tatt for seg hele utviklingsprosessen. Alt fra å kjøpe inne materiale og utstyr, programmering og kobling til å dra ut å montere. Så vi sitter igjen med en del ny kunnskap og erfaring. 

Både vi og oppdragsgiver er fornøyd med hvordan kommunikasjonen oss i mellom har vært. Vi har hatt faste møter med arbeidsgiver hvor vi har oppdatert han på hvordan vi ligger ann og han har hatt mulgihet til å komme med innspill på ting som burde gjøres anderledes. 
\subsection{Hva vi er misfornøyde med}

Som sagt er vi fornøyd med prosjektet i sin helhet, men en ting som kunne vært løst på en annen måte er hvordan ledningene fra sensorene er koblet til koblingsbrettet. Slik vi endte opp å gjøre det var at vi loddet ledningene fast til koblingsbrettet. I ettertid ser vi at dette er en tungvidt løsning dersom det skulle bli nødvendig å bytte en av sensorene. Det vi heller burde ha gjort er og tilpasset ledningene slik at de enkelt kan kobles av og på koblingsbrettet.