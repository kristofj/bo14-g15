%\cleardoublepage
\chapter{Produksjon}
\label{chap:production} 

I dette kapittelet vil vi ta for oss hvordan vi har produsert resultatene i proskjetet og beskrive selve produktet. Vi tar med hvilke verktøy vi har brukt og hvordan produksjonen foregikk. Da med tanke på hvordan vi har jobbet. Delegering av oppgaver og hvordan vi har løst problemer som har oppstått.

\section{Maskinvare}
\label{sec:HardwareDelen}

\meta{Ting å nevne: jobbet på verkstedet på skolen.. etc.. loddet sammen komponenter, bygd deksel og beskyttelse til sensorene.} \\

\subsection{Sensorer - Teknisk informasjon}

\subsubsection{SHT10}

\begin{itemize}
\item Måler temperatur og luftfuktighet.
\item Digital utverdi.
\item Operasjonsområde fra $-40\,^{\circ}\mathrm{C}$ til $120\,^{\circ}\mathrm{C}$.
\item Fargekode på koblingspunkter hvor grønn:jord, blå:data, gul:klokke og rød:VDD.
\item Spenningskilde: min 2.4V og maks 5.5V.
\end{itemize}

Ytterlig informasjon kan finnes databladet\cite{nettside:DatasheetSHT10} til SHT10. Hvordan SHT10 kommuniserer med mikrokontrolleren er beskrevet i delkapittel \ref{subsec:Ethernut} 

\subsubsection{BMP180}

\begin{itemize}
\item Måler lufttrykk.
\item Digital utverdi.
\item Operasjonsområde fra $-40\,^{\circ}\mathrm{C}$ til $85\,^{\circ}\mathrm{C}$.
\item Måler lufttrykk fra 300 til 1100 hPa. (9000 til -500m i forhold til meter over havet)
\item Koblingspunkter: VIN, 3V0, GND, SCL(data) og SDA(klokkke).
\item Klargjort for 5V logikk med integrert 3.3V regulator.
\end{itemize}

Denne versjonen av BMP180, som vi har brukt, er allerede klargjort for 5V logikk fra produsenten. Det vil si at koblingspunktet VIN er beregnet for 5V og 3V0 er beregnet for 1.6V til 3.6V. Ytterlig informasjon finnes i databladet\cite{nettside:DatasheetBMP180} til BMP180. Hvordan BMP180 kommuniserer med mikrokontrolleren er beskrevet i delkapittel \ref{subsec:Ethernut}.  


\subsubsection{WG2/O-10}

\begin{itemize}
\item Måler vindhastighet.
\item Analog utverdi 0-10V.
\item Operasjonsområdet fra $-40\,^{\circ}\mathrm{C}$ til $70\,^{\circ}\mathrm{C}$.
\item Tåler en vindstyrke på 80 m/s i opp til 30 minutter.
\item Måleområdet fra 0-50 m/s.
\item Varmeelement for å forhindre at bevegelige deler fryser fast.
\item Spenningstilførsel 13-30 V DC eller 24V AC/DC, maks 50 mA.
\item Spenningstilførsel for varmeelement 24V AC/DC, maks 20W.
\item Det er syv koblingspunkter:
	\begin{enumerate}
    \item Spenningstilførsel +
    \item Spenningstilførsel -
    \item Analog ut +
    \item Analog ut -
    \item Varmeelement +
    \item Varmeelement - 
    \item GND
    \end{enumerate}
\end{itemize}


\subsubsection{WRG2/O-10}

\begin{itemize}
\item Måler vindretning.
\item Analog utverdi 0-10V.
\item Operasjonsområdet fra $-40\,^{\circ}\mathrm{C}$ til $70\,^{\circ}\mathrm{C}$.
\item Tåler en vindstyrke på 80 m/s i opp til 30 minutter.
\item Måleområdet fra 0-50 m/s.
\item Varmeelement for å forhindre at bevegelige deler fryser fast.
\item Spenningstilførsel 13-30 V DC eller 24V AC/DC, maks 50 mA.
\item Spenningstilførsel for varmeelement 24V AC/DC, maks 20W.
\item Det er syv koblingspunkter:
	\begin{enumerate}
    \item Spenningstilførsel +
    \item Spenningstilførsel -
    \item Analog ut +
    \item Analog ut -
    \item Spenningstilførsel varmeelement +
    \item Spenningstilførsel varmeelement -
    \item GND
    \end{enumerate}
\end{itemize}



\subsection{Tilpassing for montering i felt}

Vi har to sett med sensorer som skal monteres på to forskjellige steder. Det ene skal festes til en mast ved hovedstasjonen og det andre skal festes til et tre ved sekundærstasjonen.
\subsubsection{Vindsensorene}
Som tilbehør til vindsensorene fikk vi tak i en monteringsstang fra samme produsent som er beregnet for disse sensorene. Sensorene festes og løsnes veldig enkelt fra monteringsstangen ved å skru på to låseskiver som sitter rett over lendningen ut fra sensorene. Selve monteringsstangen festes til ønsket plass ved hjelp av to metallbånd og tilhørende låsemekanisme. 

\begin{figure}[H]
  \centering
  \includegraphics[width=0.50\textwidth]{zmo40}
  \caption{Monteringsstang for vindsensorene}
\end{figure}
\subsubsection{BMP180}
Lufttrykksensoren BMP180 ble vi nødt til å montere i en koblingsboks da den ikke har noen form for beskyttelsesetui. Dette ble gjort ved å lodde fast sensoren til et koblingsbrett som ble skrudd fast inne i koblingsboksen. For å sikkre en tett kabelgjennomføring benyttet vi kabelnipler. Sensoren er også avhengig av luftgjennomstrøming i boksen for korrekt avlesning av luftrykket. Dette ble løst ved å lage et lite hull i siden og et i bunnen for drenering av vann som kan renne inn.

Festeanordningen til luftrykksensoren ved hovedstasjonen lagde vi av et bøybart metallbånd med hull i. Dette ble festet til boksen og deretter til masten ved å bøye metallbåndet rundt soltpene i masten og skru det fast med maskinskruer. Ved sekundærstasjonen ble boksen festet rett til treet ved bruk av treskruer. 

\subsubsection{SHT10}
For at temperaturmålingene ikke skal bli påvirket av direkte sollys har vi valgt å feste sensoren i en hvit koblingsboks. Koblingsboksen har tillstrekklig med åpninger for gjennomlufting slik at luften inne i boksen holder samme temperatur som på utsiden. Måten den er montert ved de to stasjonene er gjort på samme vis som ved BMP180 som er beskrevet i avsnittet ovenfor. 

\section{Bygging av værstasjonene}

Vi hadde som mål å få bygd begge værstasjonene før vi eventuelt skulle opp til Hessdalen for å montere de. Dette bygde vi på skolen sitt IT-verksted der vi hadde tilgang til alt utstyr vi skulle trenge.

\begin{figure}[H]
  \centering
  \includegraphics[width=0.75\textwidth]{kretskort1}
  \caption{Kretskort med sikkerhetskretser og kretser til sensorene}
\end{figure}


Ethernutkortet skulle være montert inne i kabinettet, mens alle sensorene skulle være utenfor. Vi brukte små bokser laget av plastikk for å beskytte temperatur, lufttrykk og luftfuktighetssensorene. Vi borret hull i boksene for å få nøyaktige temperatur, luftfuktighet og lufttrykkmålinger.

\begin{figure}[H]
  \centering
  \includegraphics[width=0.75\textwidth]{bygging1}
  \caption{Bilde av kabinettet før vi monterte på ledningene}
\end{figure}

Siden ethernutkortet ikke tåler minusgrader måtte vi isolere hele boksen. Vi hadde en ekstra sikring slik at et varmeelement ville slå inn hvis det ble mindre enn null grader i kabinettet.

Selve oppkoblingen og å sette opp alle komponentene krevdes bare litt lodding og skru fast alt med diverse skruer.


\section{Software}
\subsection{Ethernut 2.1}
\label{subsec:Ethernut}
\meta{Kan Kristoffer skrive her hvordan du jobbet, og hvilken løsninger du hadde, og litt om koden? Kan sikkert ta med eksempelkode også}

\meta{Kort intro}

\subsubsection{Implementasjon av grensesnitt mot BMP180}
BMP180 benytter I\textsuperscript{2}C/TWI for kommunikasjon med mikrokontrolleren. Dette er en egen maskinvaremodul på mikrokontrolleren, som aksesseres gjennom dataregister. Disse registrene er dokumentert i databladet til mikrokontrolleren, og vi må benytte oss av disse fire:
\begin{itemize}
\item TWI Bit Rate Register (TWBR)

Brukes sammen med TWSR for å sette klokkefrekvensen i master-modus.
\item TWI Control Register (TWCR)

Brukes for å kontrollere TWI-modulen. Her er vi interessert i bit 5, som sender START ut på bussen. Bit 4, som sender STOPP og bit 2 som starter TWI-modulen.
\item TWI Status Register (TWSR)

Bit 7-3 brukes for å sjekke status på TWI-modulen. Bit 1-0 er brukt til å skalere klokkefrekvensen.
\item TWI Data Register (TWDR)

I transmit-modus, holder dette registeret på neste byte som skal sendes. I recieve-modus vil dette registeret inneholde data som ble mottatt.
\end{itemize}

Når vi skal kalkulere lufttrykket må vi først lese av kalibreringsparametere som er lagret på sensoren. Dette er elleve 16-bits registere som befinner seg på adresse 0xAA til og med 0xBF. 

\begin{figure}[H]
	\centering
    \includegraphics[height=0.6\textwidth]{bmp180_kalibreringsparameter}
	\caption{Kalibreringsparametere på BMP180}
\end{figure}

For å lese ut disse dataene må vi først sende ut START på bussen, før vi sender adressen til hvilken enhet vi vil kommunisere med. BMP180 befinner seg på adresse 0xEE for skriving, og 0xEF for lesing. Deretter sender vi hvilken adresse vi vil lese av. BMP180 vil da svare med verdien i registeret. \cite{nettside:DatasheetBMP180}




\meta{Hvordan man bruker dataregister på mikrokontrolleren for I2C. Hvilke adresser som brukes på BMP180}
\subsubsection{Implementasjon av grensesnitt mot SHT10}
\meta{Litt om hvordan kommunikasjon foregår i detalj, ta med hvilke kommandoer som gjør hva ++}

\subsubsection{Implementasjon av ADC}
\meta{Kort om register for å styre/lese av ADC}

\subsubsection{Implementasjon av nettverk}
\meta{Hvordan nettverket blir tatt i bruk med NutOS, hvordan man henter ut tid fra NTP-server. Bruk av tråd for å sende }

\subsubsection{Implemtasjon av hovedløkke}
\meta{Hvordan hele applikasjonen er satt sammen. Skjema over hele gangen i hovedløkken.}

\subsection{Serverside}
For å ta imot TCP-pakkene med JSON-formatert data og legge det inn i databasen, var vi avhengig av å ha en server kjørende på skolens nettverk. Denne serveren valgte vi som beskrevet i (ref) å bygge på Java\cite{nettside:java}.

\subsubsection{Pakkemottak og tråder}
For å kunne motta datapakkene værstasjonen sender fra Hessdalen og ned til skolens server, bruker vi Sockets. Nærmere bestemt "TCP Stream Sockets\cite{nettside:streamsockets}". Disse tillater pålitelig to-veis kommunikasjon mellom både værstasjonen og serveren. Dette fungerer i praksis slik at serveren venter til en tilkobling er gjort, og data mottatt før den behandler data slik beskrevet under, og sender "Done" tilbake til værstasjonen. Når serveren er ferdig med dette, er den klar til å motta ny data. Dette måtte imidlertidig bli gjort trådbasert for at begge stasjonene skulle kunne sende på likt. Dermed gjorde vi det slik at så fort en ny tilkobling blir opprettet til serveren, lages det en egen tråd for å behandle denne tilkoblingen. Prosessen er den samme som beskrevet ovenfor, bortsett ifra at serveren kan håndere alle tilkoblinger selv om de kommer på likt.

\subsubsection{Konvertere JSON til SQL-spørringer}
Vi har laget en egen klasse kalt "DbManager" for å håndere den mottatte JSON-formaterte datastrengen og bygge SQL-spørringer innsetting i databasen. Her valgte vi å bruke et lite ressurskrevende bibliotek kalt Json-lib\cite{nettside:jsonlib}. Med dette biblioteket bygger vi et JSON-objekt av strengen, og kan med det lett hente ut målingsdata. Med dette bygger vi SQL-spørringer som kjøres ved hjelp av en klasse laget for å håndere databasen. Det er mer om denne prosessen i neste punkt.

\subsubsection{Databasebehandling}
Klassen "Db" har to oppgaver. Den første er å kjøre SQL-spørringene generert i "DbManager", og den andre er å returnere den autogenererte nøkkelen MySQL lager ved innsetting av data. Denne nøkkelen er svært viktig når det kommer til å få koblet riktige rader i databasen sammen og overholdt restriksjonene for fremmednøkler. Det er mer om databaseoppbygningen i (ref).

For at Java-serveren skal kunne koble seg til og håndere databasetilkoblinger, var vi avhengig av å bruke en driver fra MySQL kalt JDBC\cite{nettside:jdbc} (forkortelse for "Java Database Connectivity"). Dette er et slags bibliotek som utvider Java med muligheten for nettopp dette.   

\subsubsection{Kompilert JAR-fil}
Vi har valgt å kompilere den ferdige serveren som en JAR-fil. Dette er gjort i den hensikt å slippe å ha en relativt stor mengde klasser og biblioteker liggende løst der programmet skal kjøres fra. Ved å JAR-kompilere kan vi få med alle klasser og biblioteker i samme fil, og det blir dermed et mye mer ryddig produkt.

\subsubsection{Sikkerhet}
\meta{Åpnet port for spesifikke iper (ikke nevn hvilken port, eller hvilken ip)}

\subsubsection{Crontab}
\meta{Litt om hva, hvorfor og hvordan.}

\subsubsection{Problemer underveis}
\meta{Var det noen problemer? Hvis så, hvilke?}

\subsection{Datapresentasjon}
\meta{Kan Mikael skrive her hvordan du jobbet, og hvilken løsninger du hadde, og litt om koden? Kan sikkert ta med eksempelkode også}

\subsubsection{Backend}
\meta{Hva, hvorfor og hvordan? Stikkord: PHP, "url-variabler", JSON-konvertering, skjult databaseinfo}

\subsubsection{Biblioteker brukt}
\meta{Hvilke (jQuery, jQuery UI og Google visualisation, til hva og hvorfor?}
\begin{description}
  \item[jQuery] \hfill \\
  Tekst
  \item[jQuery UI] \hfill \\
  Tekst
  \item[Google visualisation] \hfill \\
  Tekst
\end{description}

\subsubsection{Brukervalg}
\meta{Hvilke valg har brukeren når det kommer til datafremvisning? Hvorfor? Stikkord: Idiotsikkert}

\subsubsection{Dynamisk innholdslasting}
\meta{Hva, hvorfor(fordeler) og hvordan? Stikkord: AJAX}

\subsubsection{Graffremvisning}
\meta{Hva, hvorfor og hvordan? Stikkord: Google Visualisation og begrensninger satt}

\subsubsection{Tabellvisning}
\meta{Hva, hvorfor og hvordan? Stikkord: html-tabeller}

\subsubsection{Feilhåndering}
\meta{Hvordan håndteres tidsavbrudd, ikke-eksisterende data, osv.}

\subsubsection{Design}
\meta{Hva har det vært fokus på? Hvordan har dette blitt gjort i praksis. Stikkord: enkelt/lettforståelig, oversiktlig, CSS og brukerinfo(tekstboks)}

\subsubsection{Engelsk versjon}
\meta{Hvorfor, og hvordan? Hvilke filer måtte endres}

\subsubsection{Sikkerhet}
\meta{Hvilke sikkerhetsaspekter måtte tas hensyn til, og hvordan ble de løst? Stikkord: SQL-injection}

\subsubsection{Implementasjon på hessdalen.org}
\meta{Hvordan ble dette gjort? Stikkord: CSS-konflikter}

\subsubsection{Arbeidsprosessen}
\meta{Hvilken rekkefølge ble ting gjort i? Stikkord: Hva vi begynte med, osv. Input fra arbeidsgiver}

\subsubsection{Problemer underveis}
\meta{Var det noen problemer? Hvis så, hvilke? Stikkord: Ikke all info i Google Visualisation API(ref) var helt oppdatert og mer tidkrevende enn antatt.}

