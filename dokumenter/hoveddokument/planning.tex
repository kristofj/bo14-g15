%\cleardoublepage
\chapter{Planlegging}
\label{chap:planning} 

I dette kapittelet tar vi for oss planleggingsfasen av proskjetet og hvordan vi har tenkt til å utforme produktet. Det skal også være en komplett beskrivelse av hele systemet og hvordan det fungerer. 

\section{Oppstartsfasen}

Hele bachelorprosjektet ble innledet ved at gruppen hadde et møte med veileder og oppdragsgiver. Da fikk vi vite litt mer detaljert hva vi skulle gjøre på selve oppdraget og litt om hva veilederens rolle er, gjennom prosjektperioden.


\subsection{Planlegging}

Etter jul startet vi opp med ukentlige møter med oppdragsgiver og møte med veileder hver 14. dag. Vi hadde også internmøter i gruppa 1-2 ganger i uka etter behov.\\

Etter vi hadde skrevet forprosjektrapporten, delte vi hverandre inn i forskjellige arbeidsoppgaver. Det blir mer effektivt arbeid hvis vi jobber med hver våres del, enn at alle sitter over en PC og prøver å samarbeide.



\section{Hvordan utforme produktet}


\section{Beskrivelse av hele systemet}

Målet er å lage to komplette værstasjoner som er identiske. \\

Full utstyrsliste:
\begin{itemize}
\item Ethernut mikroprosessor
\item Temperatursensor
\item Lufttrykksensor
\item Vindretningsensor
\item Vindhastighetsensor
\item Monteringsstang
\item Diverse kobberkabler
\item Strømforsynere
\item Tilgang på frigg for datalagring

\end{itemize}


////////////////////Bilde av hele systemet\\




