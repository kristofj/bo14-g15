%\cleardoublepage
\chapter{Planlegging}
\label{chap:planning} 

I dette kapittelet tar vi for oss planleggingsfasen av proskjetet og hvordan vi har tenkt til å utforme produktet. Det skal også være en komplett beskrivelse av hele systemet og hvordan det fungerer. 

\section{Planlegging}

Hele bachelorprosjektet ble innledet ved at gruppen hadde et møte med veileder og oppdragsgiver. Da fikk vi vite litt mer detaljert hva vi skulle gjøre på selve oppdraget og litt om hva veilederens rolle er, gjennom prosjektperioden.


\subsection{Oppstartsfasen}

Etter jul startet vi opp med ukentlige møter med oppdragsgiver og møte med veileder hver 14. dag. Vi hadde også internmøter i gruppa, en til to ganger i uka, etter behov.\\

Etter vi hadde skrevet forprosjektrapporten, delte vi hverandre inn i forskjellige arbeidsoppgaver. Det blir mer effektivt arbeid hvis vi jobber med hver våres del, enn at alle sitter over en PC og prøver å samarbeide.

\section {Hardware}
I dette delkapittelet vil vi forklare hvordan vi har tenkt under planleggningen av hardware.

\subsection{Sensorer}
Når det kommer til valg av sensorer vi vil benytte så er det noen kriterier å ta hensyn til: 
\begin{enumerate}
\item Må være kompatibel med $\mu$P-kortet, Ethernut 2.1
\item Tåle temperaturer helt ned til $-40\,^{\circ}\mathrm{C}$
\item Helst være ferdig utrustet for utemontering.
\end{enumerate}

Vi har satt $-40\,^{\circ}\mathrm{C}$ som den laveste temperaturen sensorene må tåle da den laveste tempteraturen i løpet av 2013 var $-34,8\,^{\circ}\mathrm{C}$ \cite{nettside:yr150313}. \\

Ut i fra disse kriteriene, nevnt ovenfor, har vi bestemt oss for disse sensorene:

\begin{figure}
        \centering
        \begin{subfigure}[H]{0.3\textwidth}
                \includegraphics[width=\textwidth]{sht10}
                \caption{SHT10 - temperatur}
                \label{fig:sht10}
                \quad
        \end{subfigure}%
        \begin{subfigure}[H]{0.25\textwidth}
                \includegraphics[width=\textwidth]{bmp180}
                \caption{BMP180 - lufttrykk}
                \label{fig:bmp180}
                \quad
        \end{subfigure}
        \\
        \begin{subfigure}[H]{0.3\textwidth}
                \includegraphics[width=\textwidth]{wg2o50}
                \caption{WG2/O50 - vindhastighet}
                \label{fig:wg2o50}
        \end{subfigure}
        \begin{subfigure}[H]{0.3\textwidth}
                \includegraphics[width=\textwidth]{wrg2o50}
                \caption{WRG2/O50 - vindretning}
                \label{fig:wrg2o50}
        \end{subfigure}
        \caption{Sensorene vi skal bruke}\label{fig:Sensorer}
\end{figure}

Disse sensorene tilfredsstiller alle kravene bortsett fra BMP180 som ikke har beskyttelse. Til denne må få tak i en innkapslingsboks som er vanntett og har tette kabelgjennomføringer. Det må også være en form for luftgjennomstrøming slik at lufttrykket blir riktig målt. Mer informasjon om disse sensorene står i delkapittel \ref{sec:HardwareDelen}.


\subsection{Kretsskjema}

\begin{figure}[H]
  \centering
  \includegraphics[width=0.60\textwidth]{SHT10}
  \caption{Kretsskjema for SHT10}
\end{figure}

\begin{figure}[H]
  \centering
  \includegraphics[width=0.60\textwidth]{BMP180}
  \caption{Kretsskjema for BMP180}
\end{figure}

BMP180 og SHT10 er digitale sensorer og disse krever en spenningstilførsel på 5V. $\mu$P-kortet, Ethernut 2.1, som vi skal bruke har egne utganger for tilførsler på 5V. Det vi trengte å vite for å koble opp kretsene mellom disse sensorene og $\mu$P-kortet stod godt beskrevet i databladet til hver sensor.

\begin{figure}[H]
  \centering
  \includegraphics[width=0.60\textwidth]{vindhastighet}
  \caption{Kretsskjema for WG2/O50}
  \label{fig:vindhastighet}
\end{figure}

\begin{figure}[H]
  \centering
  \includegraphics[width=0.60\textwidth]{vindretning}
  \caption{Kretsskjema for WRG2/O50}
  \label{fig:vindretning}
\end{figure}

I figur \ref{fig:vindhastighet} og \ref{fig:vindretning} ovenfor ser du kretskjema for de analoge vindsensorene vi har valgt ut. Disse krever en strømforsyning fra 15-30V AC/DC. Vi har valgt å bruke en strømforsyning på 24V DC da denne også kan brukes på varmefunksjonen disse sensorene også har. Disse sensorene sender ut en analog spenning på 0-10V og denne spenningen må vi strupe ned til å matche $\mu$P-kortet som ikke tåler mer enn 5V. Dette har vi løst ved å bruke en spenningsdeler som gjør at spenningen inn på $\mu$P-kortet kun varierer mellom 0-5V. 

\subsection{Kabinettet til værstasjonen}
Det er behov for et kabinett som kan beskytte $\mu$P-kortet, strømforsyningene og kretsene mellom sensorene og $\mu$P-kortet. \\ 

Viktige kriterier for valg av kabinett:
\begin{itemize}
\item Vanntett
\item Isolert mot kulde
\item Beregnet for veggmontering 
\item Må ha varmeelement og termostat
\item Tett kabelgjennomføring
\item Monteringsplate for å skru fast komponenter
\item Så liten som mulig
\end{itemize}  

\section{Software}
I dette delkapittelet vil vi forklare hvordan vi har tenkt under planleggningen av software.

\subsection{Databasen}
\meta {Forklare hvordan databasen burde settes opp.}

\subsection{Visualisering av værdata}
\meta{Forklare hvordan værdataene er tenkt å bli visualisert.}

\subsection{Serversiden som tar i mot data}
\meta{Forklare hvordan dette skal gjøres}

\subsection{Programvare til Ethernut 2.1}
\meta{Forklare hva ethenrut skal gjøre og hvordan dette kan løses}

\section{Hvordan utforme produktet}

Skriv hvordan vi skal utforme produktet


\section{Beskrivelse av hele systemet}

Målet er å lage to komplette værstasjoner som er identiske. \\


\meta{Legge til et par bilder av hele systemet her.} \\

Alle målerne er koblet til $\mu$P-kortet. Vi bruker porter på $\mu$P-kortet som er koblet til hver sin måler. Videre er portene på kortet programmert slik at vi omgjør rådataene som sensorene måler til reelle verdier som vi kan lese av. De reelle verdiene blir lagret i en database på prosject hessdalens hjemmesider, der man kan søke på diverse data som er tidligere målt.\\



\subsubsection{Full utstyrsliste:}
\begin{itemize}
\item Ethernut $\mu$Prosessor
\item Temperatursensor
\item Lufttrykksensor
\item Vindretningsensor
\item Vindhastighetsensor
\item Monteringsstang
\item Diverse kobberkabler
\item Strømforsynere
\item Tilgang på frigg for datalagring
\item Kabinett for ubeskyttede komponenter

\end{itemize}


////////////////////Bilde av hele systemet\\


Hele systemet skal være tilgjenglig for alle som er på project Hessdalens hjemmesider. 




