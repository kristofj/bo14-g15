%\cleardoublepage
\chapter{Planlegging}
\label{chap:planning} 

I dette kapittelet tar vi for oss planleggingsfasen av proskjetet og hvordan vi har tenkt til å utforme produktet. Det skal også være en komplett beskrivelse av hele systemet og hvordan det fungerer. 

\section{Planlegging}

Hele bachelorprosjektet ble innledet ved at gruppen hadde et møte med veileder og oppdragsgiver. Da fikk vi vite litt mer detaljert hva vi skulle gjøre på selve oppdraget og litt om hva veilederens rolle er, gjennom prosjektperioden.


\subsection{Oppstartsfasen}

Etter jul startet vi opp med ukentlige møter med oppdragsgiver og møte med veileder hver 14. dag. Vi hadde også internmøter i gruppa, en til to ganger i uka, etter behov.\\

Etter vi hadde skrevet forprosjektrapporten, delte vi hverandre inn i forskjellige arbeidsoppgaver. Det blir mer effektivt arbeid hvis vi jobber med hver våres del, enn at alle sitter over en PC og prøver å samarbeide.

\section{Hvordan utforme produktet}

Vi tenker å fordele arbeidsoppgavene slik at noen holder på med programmeringen mens noen andre kobler opp maskinvare. Vi kommer til å benytte oss av skolens verksted for å koble opp og montere værstasjonen.\\

Vi vil først lage et par prototyper før vi bestemmer oss for det endelige produktet. Vi vil gjerne få hele systemet til å gå før vi tenker på hva slags design og fysisk beskyttelse vi skal ha til de forskjellige komponentene.

\section {Maskinvare}
I dette delkapittelet vil vi forklare hvordan vi har tenkt under planleggningen av hardware.

\subsection{Sensorer}
Når det kommer til valg av sensorer vi vil benytte så er det noen kriterier å ta hensyn til: 
\begin{enumerate}
\item Må være kompatibel med $\mu$P-kortet, Ethernut 2.1
\item Tåle temperaturer helt ned til $-40\,^{\circ}\mathrm{C}$
\item Helst være ferdig utrustet for utemontering.
\end{enumerate}

Vi har satt $-40\,^{\circ}\mathrm{C}$ som den laveste temperaturen sensorene må tåle da den laveste tempteraturen i løpet av 2013 var $-34,8\,^{\circ}\mathrm{C}$ \cite{nettside:yr150313}. \\

Ut i fra disse kriteriene, nevnt ovenfor, har vi bestemt oss for disse sensorene:

\begin{figure}
        \centering
        \begin{subfigure}[H]{0.3\textwidth}
                \includegraphics[width=\textwidth]{sht10}
                \caption{SHT10 - temperatur}
                \label{fig:sht10}
                \quad
        \end{subfigure}%
        \begin{subfigure}[H]{0.25\textwidth}
                \includegraphics[width=\textwidth]{bmp180}
                \caption{BMP180 - lufttrykk}
                \label{fig:bmp180}
                \quad
        \end{subfigure}
        \\
        \begin{subfigure}[H]{0.3\textwidth}
                \includegraphics[width=\textwidth]{wg2o50}
                \caption{WG2/O50 - vindhastighet}
                \label{fig:wg2o50}
        \end{subfigure}
        \begin{subfigure}[H]{0.3\textwidth}
                \includegraphics[width=\textwidth]{wrg2o50}
                \caption{WRG2/O50 - vindretning}
                \label{fig:wrg2o50}
        \end{subfigure}
        \caption{Sensorene vi skal bruke}\label{fig:Sensorer}
\end{figure}

Disse sensorene tilfredsstiller alle kravene bortsett fra BMP180 som ikke har beskyttelse. Til denne må få tak i en innkapslingsboks som er vanntett og har tette kabelgjennomføringer. Det må også være en form for luftgjennomstrøming slik at lufttrykket blir riktig målt. Mer informasjon om disse sensorene står i delkapittel \ref{sec:HardwareDelen}.


\subsection{Kretsskjema}

\begin{figure}[H]
  \centering
  \includegraphics[width=0.60\textwidth]{SHT10}
  \caption{Kretsskjema for SHT10}
\end{figure}

\begin{figure}[H]
  \centering
  \includegraphics[width=0.60\textwidth]{BMP180}
  \caption{Kretsskjema for BMP180}
\end{figure}

BMP180 og SHT10 er digitale sensorer og disse krever en spenningstilførsel på 5V. $\mu$P-kortet, Ethernut 2.1, som vi skal bruke har egne utganger for tilførsler på 5V. Det vi trengte å vite for å koble opp kretsene mellom disse sensorene og $\mu$P-kortet stod godt beskrevet i databladet til hver sensor.

\begin{figure}[H]
  \centering
  \includegraphics[width=0.60\textwidth]{vindhastighet}
  \caption{Kretsskjema for WG2/O50}
  \label{fig:vindhastighet}
\end{figure}

\begin{figure}[H]
  \centering
  \includegraphics[width=0.60\textwidth]{vindretning}
  \caption{Kretsskjema for WRG2/O50}
  \label{fig:vindretning}
\end{figure}

I figur \ref{fig:vindhastighet} og \ref{fig:vindretning} ovenfor ser du kretskjema for de analoge vindsensorene vi har valgt ut. Disse krever en strømforsyning fra 15-30V AC/DC. Vi har valgt å bruke en strømforsyning på 24V DC da denne også kan brukes på varmefunksjonen disse sensorene også har. Disse sensorene sender ut en analog spenning på 0-10V og denne spenningen må vi strupe ned til å matche $\mu$P-kortet som ikke tåler mer enn 5V. Dette har vi løst ved å bruke en spenningsdeler som gjør at spenningen inn på $\mu$P-kortet kun varierer mellom 0-5V. 

\subsection{Kabinettet til værstasjonen}
Det er behov for et kabinett som kan beskytte $\mu$P-kortet, strømforsyningene og kretsene mellom sensorene og $\mu$P-kortet. \\ 

Viktige kriterier for valg av kabinett:
\begin{itemize}
\item Vanntett
\item Isolert mot kulde
\item Beregnet for veggmontering 
\item Må ha varmeelement og termostat
\item Tett kabelgjennomføring
\item Monteringsplate for å skru fast komponenter
\item Så liten som mulig
\end{itemize}

Med dette tatt i betrakning har vi valgt å gå for 

\section{Programvare}
I dette delkapittelet vil vi forklare hvordan vi har tenkt under planleggningen av programvare.

\subsection{Databasen}
Databasen som skal ta vare på værdata skal være lokalisert på en av skolens servere (Freja). Vi har valgt å bruke den populære MySQL databasen med InnoDB-motoren. Denne beslutningen er tatt på bakgrunn av at vi har god kjennskap til MySQL og InnoDB fra tidligere kurs og at det er en av verdens meste brukte åpen kildekode database(trenger kilde). Vi kommer til å overholde minst første, andre og tredje normalform fra normaliseringsreglene\cite{nettside:normalisering} i databasemodellen vår.
\begin{figure}[H]
  \centering
  \includegraphics[height=0.15\textwidth]{mysql}
  \caption{MySQL-logo}
\end{figure}

\subsection{Visualisering av værdata}
I vårt valg av hvilke teknologier vi skulle bruke for å presentere værdata på hessdalen.org, har vi tatt utangspunkt i de webprogrammeringsspråkene vi har grei kunskap om fra før. Det vil si JavaScript, PHP, og CGI-basert Python.\\

For å best kunne vurdere hva vi skulle velge, satte vi opp en liste med fordeler og ulemper ved de forskjellige metodene.
\subsubsection{CGI-basert Python\cite{nettside:python} \cite{nettside:cgi}}
Fordeler
\begin{itemize}
\item Den fulle kraften av Python implementert i nettsiden. 
\item Mange muligheter, og en stort menge biblioteker tilgjengelig.
\end{itemize}
Ulemper
\begin{itemize}
\item Hvis man ikke bruker alternative måter å kjøre CGI på, kan det fort bli treg utførelsetid på serveren. \cite{nettside:cgislow}
\item All kode kjøres også på serveren, noe som fører til høyere belastning.
\end{itemize}
\begin{figure}[H]
  \centering
  \includegraphics[height=0.15\textwidth]{python}
  \caption{Python-logo}
\end{figure}

\subsubsection{JavaScript\cite{nettside:javascript}}
Fordeler
\begin{itemize}
\item Laget for webprogrammering.
\item Stor mengde relevante bilioteker for presentering av data (Grafer, diagrammer, osv.) 
\item God samhandling med AJAX-teknologi for sømløs lasting av innhold. 
\item Utbredt språk, så det er mye god dokumentasjon. Kode kjøres av nettleser, noe som fører til mindre belastning på server.
\end{itemize}
Ulemper 
\begin{itemize}
\item All kildekode er tilgjengelig for alle som leser nettsiden. Man kan altså ikke bruke JavaScript til sensitiv informasjon.
\item Krever litt mer å debugge, men Firebug\cite{nettside:firebug} (utvidelse til Firefox og Chrome) ordner dette relativt greit.
\end{itemize}
\begin{figure}[H]
  \centering
  \includegraphics[height=0.15\textwidth]{javascript}
  \caption{JavaScript-logo}
\end{figure}

\subsubsection{PHP: Hypertext Preprocessor\cite{nettside:php}}
Fordeler
\begin{itemize}
\item Enkel syntaks og debugging.
\item Tilgivende språk (det meste fortsetter å kjøre selv om man har kodefeil).
\end{itemize}
Ulemper 
\begin{itemize}
\item Hatt mange problemer med sikkerhet i sin levetid.\cite{nettside:phpbad}
\item All kode kjøres også på serveren, noe som fører til høyere belastning.\\
\end{itemize}
\begin{figure}[H]
  \centering
  \includegraphics[height=0.15\textwidth]{php}
  \caption{PHP-logo}
\end{figure}

På bakgrunn av dette har vi bestemt oss for å presentere data ved å hovedsaklig bruke JavaScript med AJAX-teknologi og 'Google Chart'\cite{nettside:googlevis} for tegning av grafer. Som backend for å hente værdata fra databasen og konvertere til JSON\cite{nettside:json} trenger vi et språk der kildekode blir kompilert på serveren, og ikke er synlig for bruker. Til dette har vi valgt PHP ettersom det er et utbredt og relativt raskt språk til å være på serversiden.

\subsection{Serversiden som tar i mot data}
Værstasjonen vil sende data som TCP-pakker ment for en bestemt IP og port på skolens server. For å ta imot disse pakkene og sørge for at data blir lagt inn i databasen riktig, må vi lage et program som håndterer dette. Her sto vi ganske åpent på hva slags platform vi skulle bruke, men vi valgte å bygge dette på det populære språket Java av to grunner. Java versjon 1.6\cite{nettside:java} er allerede installert på skolens server der dette programmet skal kjøre. Den andre grunnen er at Java er blitt veldig populært\cite{nettside:javapopular}, og god mengde dokumentasjon er tilgjengelig på internett.\\
\begin{figure}[H]
  \centering
  \includegraphics[height=0.15\textwidth]{java}
  \caption{Java-logo}
\end{figure}
Serverprogramvaren vi skal bygge vil døgnet rundt ligge å vente på datapakker fra værstasjonen. Når den mottar en pakke med værdata vil denne informasjonen bli lagt inn i databasen med en gang. vi planlegger å bygge dette med Sockets\cite{nettside:sockets} og en objektorientert løsning der andre klasser vil konvertere data fra JSON\cite{nettside:json} til setninger som kan kjøres inn i databasen. 

\subsection{Programvare til Ethernut 2.1}
\meta{Forklare hva ethenrut skal gjøre og hvordan dette kan løses}

\section{Beskrivelse av hele systemet}

Målet er å lage to komplette værstasjoner som er identiske. \\

\meta{Legge til et par bilder av hele systemet her.} \\

Alle målerne er koblet til $\mu$P-kortet. Vi bruker porter på $\mu$P-kortet som er koblet til hver sin måler. Videre er portene på kortet programmert slik at vi omgjør rådataene som sensorene måler til reelle verdier som vi kan lese av. De reelle verdiene blir lagret i en database på prosject hessdalens hjemmesider, der man kan søke på diverse data som er tidligere målt.\\



\subsubsection{Full utstyrsliste:}
\begin{itemize}
\item Ethernut $\mu$Prosessor
\item Temperatursensor
\item Lufttrykksensor
\item Vindretningsensor
\item Vindhastighetsensor
\item Monteringsstang
\item Diverse kobberkabler
\item Strømforsynere
\item Tilgang på frigg for datalagring
\item Kabinett for ubeskyttede komponenter

\end{itemize}


////////////////////Bilde av hele systemet\\


Hele systemet skal være tilgjenglig for alle som er på project Hessdalens hjemmesider. 




