%\cleardoublepage
\chapter{Planlegging}
\label{chap:planning} 

I dette kapittelet tar vi for oss planleggingsfasen av proskjetet og hvordan vi har tenkt til å utforme produktet. Det skal også være en komplett beskrivelse av hele systemet og hvordan det fungerer. 

\section{Planlegging}

Hele bachelorprosjektet ble innledet ved at gruppen hadde et møte med veileder og oppdragsgiver. Da fikk vi vite litt mer detaljert hva vi skulle gjøre på selve oppdraget og litt om hva veilederens rolle er, gjennom prosjektperioden.


\subsection{Oppstartsfasen}

Etter jul startet vi opp med ukentlige møter med oppdragsgiver og møte med veileder hver 14. dag. Vi hadde også internmøter i gruppa, en til to ganger i uka, etter behov.

Etter vi hadde skrevet forprosjektrapporten, delte vi hverandre inn til å gjøre forskjellige arbeidsoppgaver. Det vil bli mer effektivt arbeid hvis vi jobber med hver vår del, enn at alle sitter over en PC og prøver å samarbeide.

\section{Hvordan utforme produktet}

Vi tenker å fordele arbeidsoppgavene slik at noen holder på med programmeringen mens noen andre kobler opp maskinvare. Vi kommer til å benytte oss av skolens verksted for å koble opp og montere værstasjonen.

Vi vil først lage et par prototyper før vi bestemmer oss for det endelige produktet. Vi vil gjerne få hele systemet til å gå før vi tenker på hva slags design og fysisk beskyttelse vi skal ha til de forskjellige komponentene.

\section {Maskinvare}
I dette delkapittelet vil vi forklare hvordan vi har tenkt under planleggningen av hardware.

\subsection{Sensorer}
Når det kommer til valg av sensorer vi vil benytte så er det noen kriterier å ta hensyn til: 
\begin{itemize}
\item Må være kompatibel med mikrokontrolleren.
\item Tåle temperaturer helt ned til $-40\,^{\circ}\mathrm{C}$.
\item Helst være ferdig utrustet for utemontering.
\end{itemize}
Vi har satt $-40\,^{\circ}\mathrm{C}$ som den laveste temperaturen sensorene må tåle da den laveste tempteraturen i løpet av 2013 var $-34,8\,^{\circ}\mathrm{C}$ \cite{nettside:yr150313}.
\newline
\newline
Ut i fra disse kriteriene, nevnt ovenfor, har vi bestemt oss for disse sensorene:

\begin{figure}[H]
        \centering
        \begin{subfigure}[H]{0.3\textwidth}
                \includegraphics[width=\textwidth]{sht10}
                \caption{SHT10 - temperatur og luftfuktighet}
                \label{fig:sht10}
                \quad
        \end{subfigure}%
        \begin{subfigure}[H]{0.25\textwidth}
                \includegraphics[width=\textwidth]{bmp180}
                \caption{BMP180 - lufttrykk}
                \label{fig:bmp180}
                \quad
        \end{subfigure}
        \\
        \begin{subfigure}[H]{0.3\textwidth}
                \includegraphics[width=\textwidth]{wg2o50}
                \caption{WG2/O50 - vindhastighet}
                \label{fig:wg2o50}
        \end{subfigure}
        \begin{subfigure}[H]{0.3\textwidth}
                \includegraphics[width=\textwidth]{wrg2o50}
                \caption{WRG2/O50 - vindretning}
                \label{fig:wrg2o50}
        \end{subfigure}
        \caption{Sensorene vi skal bruke}\label{fig:Sensorer}
\end{figure}
Disse sensorene tilfredsstiller alle kravene, bortsett fra BMP180 som ikke har beskyttelse. Til denne må få tak i en innkapslingsboks som er vanntett og har tette kabelgjennomføringer. Det må også være en form for luftgjennomstrøming slik at lufttrykket blir riktig målt. Mer informasjon om disse sensorene står i delkapittel \ref{sec:HardwareDelen}.

BMP180\cite{nettside:lufttrykk}, SHT10\cite{nettside:temperatur}, WRG2/O50\cite{nettside:vindretning} og WG2/=50\cite{nettside:vindhastighet}.


\subsection{Kretsskjema}

\begin{figure}[H]
  \centering
  \includegraphics[width=0.60\textwidth]{SHT10}
  \caption{Kretsskjema for SHT10}
\end{figure}

\begin{figure}[H]
  \centering
  \includegraphics[width=0.60\textwidth]{BMP180}
  \caption{Kretsskjema for BMP180}
\end{figure}

BMP180 og SHT10 er digitale sensorer og disse krever en spenningstilførsel på 5V. Mikrokontrolleren  vi skal bruke har egne utganger for tilførsler på 5V. Det vi trengte å vite for å koble opp kretsene mellom disse sensorene og mikrokontrolleren stod godt beskrevet i databladene til sensorene.

\begin{figure}[H]
  \centering
  \includegraphics[width=0.60\textwidth]{vindhastighet}
  \caption{Kretsskjema for WG2/O50}
  \label{fig:vindhastighet}
\end{figure}

\begin{figure}[H]
  \centering
  \includegraphics[width=0.60\textwidth]{vindretning}
  \caption{Kretsskjema for WRG2/O50}
  \label{fig:vindretning}
\end{figure}

I figur \ref{fig:vindhastighet} og \ref{fig:vindretning} ovenfor ser du kretskjema for de analoge vindsensorene vi har valgt ut. Disse krever en strømforsyning fra 15-30V AC/DC. Vi har valgt å bruke en strømforsyning på 24V DC da denne også kan brukes på varmefunksjonen disse sensorene også har. Disse sensorene sender ut en analog spenning på 0-10V og denne spenningen må vi strupe ned til å matche mikrokontrolleren som ikke tåler mer enn 5V. Dette har vi løst ved å bruke en spenningsdeler som gjør at spenningen inn på mikrokontrolleren kun varierer mellom 0-5V. 

\subsection{Kabinettet til værstasjonen}
Det er behov for et kabinett som kan beskytte mikrokontrolleren, strømforsyningene og kretsene mellom sensorene og mikrokontrolleren. \\ 

Viktige kriterier for valg av kabinett:
\begin{itemize}
\item Vanntett
\item Isolert mot kulde
\item Beregnet for veggmontering 
\item Må ha varmeelement og termostat
\item Tett kabelgjennomføring
\item Monteringsplate for å skru fast komponenter
\item Så liten som mulig
\end{itemize}

Med dette tatt i betrakning har vi valgt å gå for et kabinett i metall med monteringsplate i stål og en dør som er fulgt forseglet. 

\begin{figure}[H]
  \centering
  \includegraphics[width=0.25\textwidth]{kabinett}
  \caption{Kabinett med dør}
  \label{fig:kabinett}
\end{figure}

Dette skapet har en beskyttelsesgrad på IP 66, \textit{International Protection Marking}\cite{nettside:IP}, hvilket vil si at det kvalifiserer meget godt for utendørsmontering. Eneste vi må forbedre er isolasjon mot varme og kulde. Dette kan løses ved å bruke isopor innvendig, da det hverken leder kulde eller varme. I tillegg til etterisolasjon skal vi også montere et termostsatstyrt varmeelement for å være helt sikker på at det ikke oppstår kondens eller at ting fryser inne i kabinettet.  

\section{Programvare}
I dette delkapittelet vil vi forklare hvordan vi har tenkt under planleggningen av programvare.

\subsection{Databasen}
Databasen som skal ta vare på værdata skal være lokalisert på en av skolens servere (Freja). Vi har valgt å bruke den populære MySQL databasen med InnoDB-motoren. Denne beslutningen er tatt på bakgrunn av at vi har god kjennskap til MySQL og InnoDB fra tidligere kurs og at det er en av verdens meste brukte åpen kildekode database(trenger kilde). Vi kommer til å overholde minst første, andre og tredje normalform fra normaliseringsreglene\cite{nettside:normalisering} i databasemodellen vår.
\begin{figure}[H]
  \centering
  \includegraphics[height=0.15\textwidth]{mysql}
  \caption{MySQL-logo}
\end{figure}

\subsection{Visualisering av værdata}
I vårt valg av hvilke teknologier vi skulle bruke for å presentere værdata på hessdalen.org, har vi tatt utangspunkt i de webprogrammeringsspråkene vi har grei kunskap om fra før. Det vil si JavaScript, PHP, og CGI-basert Python.

For å best kunne vurdere hva vi skulle velge, satte vi opp en liste med fordeler og ulemper ved de forskjellige metodene.
\subsubsection{CGI-basert Python\cite{nettside:python} \cite{nettside:cgi}}
Fordeler
\begin{itemize}
\item Den fulle kraften av Python implementert i nettsiden. 
\item Mange muligheter, og en stort menge biblioteker tilgjengelig.
\end{itemize}
Ulemper
\begin{itemize}
\item Hvis man ikke bruker alternative måter å kjøre CGI på, kan det fort bli treg utførelsetid på serveren. \cite{nettside:cgislow}
\item All kode kjøres også på serveren, noe som fører til høyere belastning.
\end{itemize}
\begin{figure}[H]
  \centering
  \includegraphics[height=0.15\textwidth]{python}
  \caption{Python-logo}
\end{figure}

\subsubsection{JavaScript\cite{nettside:javascript}}
Fordeler
\begin{itemize}
\item Laget for webprogrammering.
\item Stor mengde relevante bilioteker for presentering av data (Grafer, diagrammer, osv.) 
\item God samhandling med AJAX-teknologi for sømløs lasting av innhold. 
\item Utbredt språk, så det er mye god dokumentasjon. Kode kjøres av nettleser, noe som fører til mindre belastning på server.
\end{itemize}
Ulemper 
\begin{itemize}
\item All kildekode er tilgjengelig for alle som leser nettsiden. Man kan altså ikke bruke JavaScript til sensitiv informasjon.
\item Krever litt mer å debugge, men Firebug\cite{nettside:firebug} (utvidelse til Firefox og Chrome) ordner dette relativt greit.
\end{itemize}
\begin{figure}[H]
  \centering
  \includegraphics[height=0.15\textwidth]{javascript}
  \caption{JavaScript-logo}
\end{figure}

\subsubsection{PHP: Hypertext Preprocessor\cite{nettside:php}}
Fordeler
\begin{itemize}
\item Enkel syntaks og debugging.
\item Tilgivende språk (det meste fortsetter å kjøre selv om man har kodefeil).
\end{itemize}
Ulemper 
\begin{itemize}
\item Hatt mange problemer med sikkerhet i sin levetid.\cite{nettside:phpbad}
\item All kode kjøres også på serveren, noe som fører til høyere belastning.\\
\end{itemize}
\begin{figure}[H]
  \centering
  \includegraphics[height=0.15\textwidth]{php}
  \caption{PHP-logo}
\end{figure}

På bakgrunn av dette har vi bestemt oss for å presentere data ved å hovedsaklig bruke JavaScript med AJAX-teknologi og 'Google Chart'\cite{nettside:googlevis} for tegning av grafer. Som backend for å hente værdata fra databasen og konvertere til JSON\cite{nettside:json} trenger vi et språk der kildekode blir kompilert på serveren, og ikke er synlig for bruker. Til dette har vi valgt PHP ettersom det er et utbredt og relativt raskt språk til å være på serversiden.

\subsection{Serversiden}
Værstasjonen vil sende data som TCP-pakker ment for en bestemt IP og port på skolens server. For å ta imot disse pakkene og sørge for at data blir lagt inn i databasen riktig, må vi lage et program som håndterer dette. Her sto vi ganske åpent på hva slags platform vi skulle bruke, men vi valgte å bygge dette på det populære språket Java av to grunner. Java versjon 1.6\cite{nettside:java} er allerede installert på skolens server der dette programmet skal kjøre. Den andre grunnen er at Java er blitt veldig populært\cite{nettside:javapopular}, og god mengde dokumentasjon er tilgjengelig på internett.\\
\begin{figure}[H]
  \centering
  \includegraphics[height=0.15\textwidth]{java}
  \caption{Java-logo}
\end{figure}
Serverprogramvaren vi skal bygge vil døgnet rundt ligge å vente på datapakker fra værstasjonen. Når den mottar en pakke med værdata vil denne informasjonen bli lagt inn i databasen med en gang. vi planlegger å bygge dette med Sockets\cite{nettside:sockets} og en objektorientert løsning der andre klasser vil konvertere data fra JSON\cite{nettside:json} til setninger som kan kjøres inn i databasen. 

\subsection{Programvare på Ethernut 2.1}

Mikrokontrolleren Ethernut 2.1 skal spille en sentral rolle i systemet. Det er dens oppgave å 
lese av sensorene, og sende målingene til server for permanent lagring.
På mikrokontrolleren kommer vi til å benytte det lille sanntidsoperativsystemet NutOS, som har blant annet implementert TCP/IP-stacken. Vår applikasjon kommer til å bli skrevet i språket C.\\

Oppdragsgiver hadde følgende krav til funksjonaliteten til mikrokontrolleren:
\begin{itemize}
\item Mikrokontrolleren skal gjøre ti målinger hvert femte minutt, og skal sende data for den siste timen til server. Fra disse målingene skal den høyeste og laveste verdi fjernes før det regnes ut gjennomsnitt. I tilfellet støy skulle ødelegge for målingene.
\item Watchdog skal benyttes slik at mikrokontrolleren starter om hvis noe skulle gå galt, eller prosessoren skulle henge seg.
\item Klokken på mikrokontrolleren skal stille seg selv ved å kontakte en server som kjører NTP (Network Time Protocol). Her har vi valgt å benytte NTP Pool Project sin server i Norge.
\end{itemize}

\subsection{Grensesnitt mot sensorene}
\subsubsection{BMP180}
BMP180 benytter den kjente kommunikasjonsbusen I\textsuperscript{2}C (Inter-Integrated Circuit), også kalt TWI (Two-wire Serial Interface). Et TWI maskinvaregrensesnitt er tilgjengelig på mikrokontrolleren.
I\textsuperscript{2}C er en flerbruker seriell databuss som benytter seg av et master-slave system. Vår implementasjon krever kun en master, Ethernut 2.1, men alle enheter har mulighet til å være master i en slik buss. Alle enhetene kobles til to kabler, som er SCL (klokken) og SDA (data). Klokken styres av master når det foregår kommunikasjon. Hver enhet koblet til bussen har en adresse på 8 bit. Den siste biten bestemmer om man ønsker å skrive til, eller lese av sensoren.

% TODO: Legge inn et bilde av en typisk I2C-buss.
Det er fire forskjellige moduser for hver enhet tilkoblet bussen:
\begin{itemize}
\item Master transmit, hvor master sender data til slave.
\item Master recieve, hvor master mottar data fra slave.
\item Slave transmit, hvor slave sender data til master.
\item Slave recieve, hvor slave mottar data fra master.
\end{itemize}

Typisk kommunikasjon mellom to enheter koblet til bussen starter med at en master sender et START-signal. Da er master i transmit-modus, og etterfulges med å sende en 7-bit adresse, som er adressen til enheten den ønsker å kommunisere med.
Hvis slaven eksisterer svarer enheten med en acknowledge(ACK), master går da inn i transmit eller recieve modus avhengig av hva master vil gjøre. Hver databyte sendt vil da bli svart på med en ACK, unntatt på siste byte.
Etter ferdig sending, kan master igjen velge å sende et START-signal hvis den ønsker å kommunisere mer. Master avslutter kommunikasjon ved å sende et STOP-signal.

Når mikrokontrollen skal kommunisere med BMP180 må den først lese av kalibreringsparametere som er lagret på sensoren. Deretter må sensoren få beskjed om å måle lufttrykket, før mikrokontrolleren leser av råverdien. Mikrokontrolleren må da bruke både råverdien og kalibreringsparameterene til å kalkulere lufttrykket.
\subsubsection{SHT10}
SHT10 benytter sin egen kommunikasjonsprotokol som er relativt lik I\textsuperscript{2}C. Protokollen er  dokumentert i sensorens datablad. Kommunikasjon med sensoren må implementeres ved å bruke såkalt bit banging. Dette er en teknikk for seriekommunikasjon. Som går ut på å manuelt veksle de digitale utgangene i programvare, istedet for å bruke dedikert maskinvare. Pågrunn av mangel på dedikert maskinvare vil dette påvirke ytelsen, prosessoren må gjøre all jobb ved kommunikasjon. SHT10 må det styres med to utganger, SCK for klokke og DATA for data.

Mikrokontrolleren må først sende en kommando til sensoren som bestemmer hva som skal måles. Før sensoren sender råverdien til mikrokontrolleren. Denne råverdien brukes sammen med kalibreringsparametere, som er dokumentert i datablad, for å regne ut temperatur og luftfuktighet.
\subsubsection{Vindretning- og hastighet}
Vindretning- og hastighetsensorene er analoge. Disse vil vi lese av ved hjelp av mikrokontrollerens ADC (Analog to Digital Converter). Dette er en maskinvare-modul som er en del av mikrokontrolleren. ADC-en er dokumentert i mikrokontrollerens datablad. Her må vi kun lese av verdien på inngangene og konvertere de til respektive verdier.

\section{Beskrivelse av hele systemet}

\begin{figure}[H]
  \centering
  \includegraphics[height=0.60\textwidth]{Oversiktsskjema}
  \caption{Systemoversikt}
  \label{fig:Systemoversikt}
\end{figure}

I figur \ref{fig:Systemoversikt} ser du en oversikt over systemet og hvilke komponenter som kommuniserer med hverandre. Mikrokontrolleren leser av sensorene, behandler dataene og sender disse til serveren. På serveren blir dataene lagret i en database. Disse dataene blir hentet ut og visualisert på Project Hessdalen sin nettside.   



