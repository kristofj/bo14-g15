\documentclass[12pt,a4paper]{article}
\usepackage[utf8]{inputenc}
\usepackage{paralist}

%Dokumentet lagres i formatet dd_mm_yy.tex

\begin{document}
\title{
Møtereferat\\
Gruppe 15
}
\author{\Large Mikael Grimstad}
\date{Mumblemøte(VOIP) 08.01.14 kl. 19:00}
\maketitle

\section*{Deltagere}
\begin{compactitem}
	\item Mikael
	\item Kristoffer
	\item Kristian
	\item Morten
\end{compactitem}
\section*{Saker}
	\subsection*{Hvor skal vi lagre filer?}
	Prøver oss på å bruke GitHub til alt av filer. Ingen er erfarne med hvordan det brukes, men vi lærer fort.
	\subsection*{Hvordan skal skriving av referater, osv. forgå?}
	Sekretærrollen går på rundgang ukentlig. Mikael begynner i uke 3, deretter blir syklusen Kristian, Kristoffer og Morten.
	\subsection*{Når skal vi avtale neste møte med veileder?}
	Sender mail og spør om møte mandag 13. januar.
	\subsection*{Hva skal vi spørre veileder om?}
	Er det viktig med arbeidslogg? Generelle anbefalinger og deling av erfaringer.
	\subsection*{Hvordan skal vi holde styr på når vi skal ha møter?}
	Setter opp en Google kalender for dette.
\end{document}