\documentclass[12pt,a4paper]{article}
\usepackage[utf8]{inputenc}
\usepackage{paralist}

%Dokumentet lagres i formatet dd_mm_yy.tex

\begin{document}
\title{Møtereferat}
\author{\Large Av Mikael Grimstad}
\date{Remmen, Halden 10.02.14 kl. 14:00}
\maketitle

\subsection*{Deltagere}
\begin{compactitem}
	\item Mikael
	\item Morten
	\item Kristoffer
	\item Kristian
	\item Einar
	\item Erling
\end{compactitem}
\subsection*{Saker}
\begin{compactitem}
	\item Først møte med Einar (veileder). Der vi diskuterte oppgavens løp med han. Blitt enige om at vi må komme i gang med rapportskriving snart, og at vi faktisk må legge litt arbeid i designet på nettsiden der data skal presenteres. Nytt møte med Einar om 14 dager.
	\item Deretter møte med Erling (oppdragsgiver). Motstander og kondensatorer kan vi få på skolen. Mulighet for å bruke robotlab til å jobbe på og eventuelt sette fra oss arbeidet på. Studentskap kan også brukes til lagring. Kan være vi må bruke lavpassfilter til å rense signalene fra analoge sensorer. Bestilling av det fra adafruit blir lagt inn idag, men det fra den tyske siden er ikke på lager enda. Trenger en metallkasse til å legge utstyret i. Sjekke ut av watchdog. Lagre maks, min, middel og nåværende på temperatur. Vindhastighet bør vi vurdere noe av det samme (for å få med oss vindkast, osv.) 
\end{compactitem}
\end{document}